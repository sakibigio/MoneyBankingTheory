\subsection{Leontief Case}

For the Leontief matching function $G(a,b) = \min\{a,b\}$, we have $\gamma(\theta) = \min\{1,\theta\}$.

\textbf{Case 1: $\theta_0 > 1$}. By Lemma \ref{P_ContinuousProbabilities}, ODE for market tightness becomes:
\[
\dot{\theta}_{\tau} = -\bar{\lambda}\theta_{\tau}[\theta_{\tau}^{-1} - 1] = \bar{\lambda}(\theta_{\tau} - 1).
\]

Solving this linear ODE with initial condition $\theta_0$:
\[
\theta_{\tau} = 1 + (\theta_0 - 1)e^{\bar{\lambda}\tau}
\]
The matching intensities are:
\begin{align*}
\psi^+_{\tau} &= \bar{\lambda}\gamma(\theta_{\tau}) = \bar{\lambda}\\
\psi^-_{\tau} &= \bar{\lambda}\gamma(\theta_{\tau}^{-1}) = \frac{\bar{\lambda}}{\theta_{\tau}} = \frac{\bar{\lambda}}{1 + (\theta_0 - 1)e^{\bar{\lambda}\tau}}.
\end{align*}

The matching probability, $\Psi^+$, is thus:
\[
\Psi^+ = 1 - \exp\left(-\int_0^1 \bar{\lambda}d\tau\right) = 1 - e^{-\bar{\lambda}}.
\]
For $\Psi^-$, we use that $\Psi^-=\Psi^+/\theta_0$ to conclude that:
\[
\Psi^-= \frac{1 - e^{-\bar{\lambda}}}{\theta_0}.
\]
We can verify that the same conclusion is reached by evaluating the integral: $\Psi^-=1-\exp(-\int^1_0\psi^{-}_\tau d\tau).$ 

\textbf{Case 2: $\theta_0 < 1$.} By symmetry (Proposition \ref{prop:symmetry}), we have that:
\[
\theta_{\tau}^{-1} = 1+(\theta_{0}^{-1}-1)e^{\bar{\lambda}\tau} \Leftrightarrow
\theta_{\tau}= \frac{\theta_0}{\theta_0 + (1-\theta_0)e^{\bar{\lambda}\tau}}.
\]
And, also, by symmetry:
\[
\Psi^+ = \theta_0(1-e^{-\bar{\lambda}}), \quad \Psi^- = 1-e^{-\bar{\lambda}}.
\]

\subsection{Cobb-Douglas Case}

For the Cobb-Douglas matching function $G(a,b) = a^{1/2}b^{1/2}$, we have $\gamma(\theta) = \theta^{1/2}$. The ODE for tightness specializes to:
\[
\dot{\theta}_{\tau} = -\bar{\lambda}\theta_{\tau}[\theta_{\tau}^{-1/2} - \theta_{\tau}^{1/2}] = -\bar{\lambda}\theta_{\tau}^{1/2}(1 - \theta_{\tau}).
\]

We solve the ODE by change of variables: $u \equiv \theta_{\tau}^{1/2}$. Then $\dot{\theta}_{\tau} = 2u\dot{u}$, giving:
\[
2u\dot{u} = -\bar{\lambda}u(1-u^2)
\Rightarrow
\dot{u} = -\frac{\bar{\lambda}}{2}(1-u^2)
\]
Separating variables:
\[
\frac{du}{1-u^2} = -\frac{\bar{\lambda}}{2}d\tau.
\]
Using partial fractions, we write:
\[
\frac{1}{1-u^2} = \frac{1}{2}\left(\frac{1}{1+u} + \frac{1}{1-u}\right),
\]
Thus, integrating both sides of the relationship above:
\[
\frac{1}{2}\ln\left(\frac{1+u}{1-u}\right) = -\frac{\bar{\lambda}\tau}{2} + C.
\]
Since, the initial condition $u(0) = \sqrt{\theta_0}$, we obtain:
\[
\ln\left(\frac{1+u}{1-u}\right) = -\bar{\lambda}\tau + \ln\left(\frac{1+\sqrt{\theta_0}}{1-\sqrt{\theta_0}}\right)
\]
We clear out $u$ taking the exponential on both sides:
\[
\frac{1+u_\tau}{1-u_\tau} = \frac{1+\sqrt{\theta_0}}{1-\sqrt{\theta_0}}e^{-\bar{\lambda}\tau}.
\]
Therefore:
\[
u_{\tau} = \frac{(1+\sqrt{\theta_0})e^{-\bar{\lambda}\tau} - (1-\sqrt{\theta_0})}{(1+\sqrt{\theta_0})e^{-\bar{\lambda}\tau} + (1-\sqrt{\theta_0})}
\Rightarrow
\theta_{\tau} = u_{\tau}^2 = \left(\frac{(1+\sqrt{\theta_0})e^{-\bar{\lambda}\tau} - (1-\sqrt{\theta_0})}{(1+\sqrt{\theta_0})e^{-\bar{\lambda}\tau} + (1-\sqrt{\theta_0})}\right)^2.
\]
We can verify that for $\tau=0$ the initial condition holds. 

\textbf{Stopping time:} For the Cobb-Douglas case, the asymptotic limit $\bar{\gamma}$ is unbounded. Thus, trade can vanish in finite time. The ODE is valid locally as long as trade occurs. Next, we obtain the value of time at which the trade stops. 

 Trade stops when $\theta_{\tau} = 0$ which can be reached if $\theta_0 < 1$. Likewise, trade stops at the time where $\theta_{\tau} = 0$ asymptotes to infinity, (a case only relevant when $\theta_0 > 1$). 

 When $\theta_0<1$, note that the denominator in the formula $\theta_{\tau}$ is always positive. Thus, the numerator reaches zero at:
 $T = \frac{1}{\bar{\lambda}}\ln\left(\frac{1+\sqrt{\theta_0}}{1-\sqrt{\theta_0}}\right)$. When $\theta_0>1$, the numerator is always positive, but the denominator reaches zero at $T = \frac{1}{\bar{\lambda}}\ln\left(\frac{1+\sqrt{\theta_0}}{\sqrt{\theta_0}-1}\right).$ Thus, the stopping time in either case occurs when:
\[
T = \min\left\{ \frac{1}{\bar{\lambda}}\ln\left|\frac{1+\sqrt{\theta_0}}{1-\sqrt{\theta_0}}\right|,1\right\}.
\]
The effective stopping time is $\min\{T, 1\}$.

\textbf{Matching probabilities:} Matching probabilities follow from:
\[
\Psi^{\pm} = 1 - \exp\left(-\int_0^T \psi^{\pm}_s ds\right).
\]
where $\psi^+_{\tau} = \bar{\lambda}\sqrt{\theta_{\tau}}=\bar{\lambda}u_{\tau}$ and $\psi^-_{\tau} = \bar{\lambda}/\sqrt{\theta_{\tau}}=\bar{\lambda}u^{-1}_{\tau}$. We proceed with some calculations:
\[
\int_0^T \psi^{\pm}d\tau=\int_0^T \bar{\lambda}u_s ds. 
\]
Let $\alpha =\frac{ 1+\sqrt{\theta_0}}{ 1-\sqrt{\theta_0}}.$ Then:
\[
u_s = \frac{\alpha e^{-\bar{\lambda}s} - 1}{\alpha e^{-\bar{\lambda}s} + 1}.
\]
To integrate, note that:
$u_s = \frac{1}{\bar{\lambda}}\frac{\partial [-\bar{\lambda}s-2\ln({\alpha e^{-\bar{\lambda}s} + 1)}]}{\partial s}=\frac{1}{\bar{\lambda}}(-\bar{\lambda}+\frac{2\bar{\lambda}e^{-\bar{\lambda}s}}{\alpha e^{-\bar{\lambda}s} + 1}).$
Therefore:
\[
\int_0^T-\bar{\lambda} u_s ds = \int_0^T \frac{\partial [\bar{\lambda}s+2\ln({\alpha e^{-\bar{\lambda}s} + 1)}]}{\partial s} ds=  \bar{\lambda}s+2\ln(\alpha e^{-\bar{\lambda}s}+1)|^{s=T}_{s=0}.
\]
Thus, 
\[
\exp\left(-\int_0^T \psi^{-}_s ds\right)=\exp\left(\bar{\lambda}\min\{T,1\}+2\ln\left(\frac{\alpha e^{-\bar{\lambda}s}+1}{\alpha+1}\right)\right).
\]
Thus, we have that:
\[
\Psi^{+}=1-\exp(-\bar{\lambda}T)\left(\frac{\alpha +e^{-\bar{\lambda}T}}{\alpha+1}\right)^2.
\]
Substituting $\alpha$, we get the formula in the table. 

The formula for $\Psi^-$ is obtained using $\Psi^-=\Psi^{+}/\theta_0=\Psi^{+}/u^2_0$:
\[
\Psi^- = \frac{\Psi^+}{\theta_0} = \left(\frac{\alpha+1}{\alpha-1}\right)^2 \Psi^+
= \left(\frac{\alpha+1}{\alpha-1}\right)^2 - \exp(-\bar{\lambda}T)\left(\frac{\alpha e^{-\bar{\lambda}T}+1}{\alpha-1}\right)^2.
\]
Substituting back $\alpha = \frac{1+\sqrt{\theta_0}}{1-\sqrt{\theta_0}}$:
\begin{align*}
\Psi^- &= \left(\frac{2\sqrt{\theta_0}}{2}\right)^2\left(\frac{1}{\theta_0}\right) - \exp(-\bar{\lambda}T)\left(\frac{(1+\sqrt{\theta_0})e^{-\bar{\lambda}T} + (1-\sqrt{\theta_0})}{2\sqrt{\theta_0}}\right)^2\left(\frac{1}{\theta_0}\right)\\
&= 1 - \exp(-\bar{\lambda}T)\left(\frac{(1+\sqrt{\theta_0})e^{-\bar{\lambda}T} - (1-\sqrt{\theta_0})e^{\bar{\lambda}T}}{2\sqrt{\theta_0}}\right)^2\\
&= 1 - \exp(-\bar{\lambda}T)\left(\frac{(1+\sqrt{\theta_0}) - (1-\sqrt{\theta_0})e^{\bar{\lambda}T}}{(1+\sqrt{\theta_0}) - (1-\sqrt{\theta_0})}\right)^2.
\end{align*}
This completes the derivation of both $\Psi^+$ and $\Psi^-$ for the Cobb-Douglas case.

