\usepackage{cochineal}
\renewcommand{\familydefault}{\rmdefault}
\usepackage[T1]{fontenc}
%\usepackage[latin9]{inputenc}
\usepackage[active]{srcltx}
\usepackage{float}
\usepackage{amsmath}
\usepackage{amsthm}
\usepackage{amssymb}
\usepackage{graphicx}
\usepackage{geometry}
\geometry{verbose,tmargin=1.in,bmargin=1.in,lmargin=1in,rmargin=1in}
\usepackage{setspace}
\usepackage[authoryear,longnamesfirst]{natbib}

\usepackage{pdflscape}
\usepackage{placeins}


\onehalfspacing

\makeatletter

%%%%%%%%%%%%%%%%%%%%%%%%%%%%%% LyX specific LaTeX commands.
%% Because html converters don't know tabularnewline
\providecommand{\tabularnewline}{\\}

%%%%%%%%%%%%%%%%%%%%%%%%%%%%%% Textclass specific LaTeX commands.
\theoremstyle{plain}
\newtheorem{cor}{\protect\corollaryname}
\newenvironment{lyxlist}[1]
	{\begin{list}{}
		{\settowidth{\labelwidth}{#1}
		 \setlength{\leftmargin}{\labelwidth}
		 \addtolength{\leftmargin}{\labelsep}
		 \renewcommand{\makelabel}[1]{##1\hfil}}}
	{\end{list}}

%%%%%%%%%%%%%%%%%%%%%%%%%%%%%% User specified LaTeX commands.
\special{papersize=8.5in,11in}
%\usepackage{palatino,amssymb,amsfonts,amsmath,latexsym,setspace}
\usepackage{amsfonts}
\usepackage{latexsym}
% \usepackage[dvips]{graphicx}
\usepackage{amsmath}
\usepackage{fancyhdr}

%%%%%%%%%%%%%%%%%%%%%%%%%%%%%%%%%%%%%%%%%%%%%%%%%%%%%%%%%%%%%%%%%%%%%%%%%%%%%%%%%%%%%%%%%%%%%%%%%%%%%%%%%%%%%%%%%%%%%%%%%%%%%%%%%%%%%%%%%%%%%%%%%%%%%%%%%%%%%%%%%%%%%%%%%%%%%%%%%%%%%%%%%%%%%%%%%%%%%%%%%%%%%%%%%%%%%%%%%%%%%%%%%%%%%%%%%%%%%%%%%%%%%%%%%%%%
\usepackage{amsfonts}
\usepackage{array}
\usepackage{lscape}
% \usepackage{color}

\usepackage{colortbl}
 \usepackage{xcolor}
\usepackage{caption}
\usepackage{subfig}
% \usepackage{subcaption}
\usepackage{sidecap}

\usepackage{tikz,pgf}
\usetikzlibrary{decorations.text}
\usetikzlibrary{patterns}
\usepackage{pgfplots}
% \pgfplotsset{compat=1.18}
\usepackage{verbatim}
\usepackage{algorithm}
\usepackage{algorithmic}

% \usepackage[longnamesfirst]{natbib}
\bibliographystyle{aer}

\usepackage{hyperref}
\hypersetup{breaklinks = true,
colorlinks=true, anchorcolor= ChadBlue, citecolor= ChadBlue,
filecolor= ChadBlue, linkcolor= ChadBlue, menucolor= ChadBlue,
urlcolor= blue, citebordercolor= 1 0 0, menubordercolor=1 0
0, urlbordercolor=1 0 0, runbordercolor=1 0 0 }


\setcounter{MaxMatrixCols}{10}
%  Chad's style file for nice paper format.
%  Various ideas for this style are taken from
%    -- Matthias Doepke and Michele Tertilt
%    -- Tony Roberts


% example file to change the style of LaTeX
% first colour for latex or pdflatex
%\ifx\pdfoutput\@undefined\usepackage[usenames,dvips]{color}
% \else\usepackage[usenames,dvipsnames]{color}
% and fix pdf colour problems
% \IfFileExists{pdfcolmk.sty}{\usepackage{pdfcolmk}}{} 
%\fi

% Chad's colors:
% See http://web.njit.edu/~kevin/rgb.txt.html for possibilities
\definecolor{ChadDarkBlue}{rgb}{.1,0,.2}  
\definecolor{ChadBlue}{rgb}{.1,.1,.5}  
\definecolor{ChadRoyal}{rgb}{.2,.2,.8}  
%\definecolor{ChadGreen}{rgb}{0,.35,.1}
%\definecolor{ChadGreen}{rgb}{0,.5,.25}  % Too bright
%\definecolor{ChadGreen}{rgb}{0,.4,.2}    % Still too bright
\definecolor{ChadGreen}{rgb}{0,.4,0}    % Dark Green
%\definecolor{ChadRed}{rgb}{.8,.1,.2}    % Too bright
\definecolor{ChadRed}{rgb}{.5,0,.5}  % purple
\definecolor{ChadMagenta}{rgb}{.6,0,.6}  % magenta
\definecolor{webbrown}{rgb}{.1,0,.2}  % magenta

%%% HYPERLINKS %%%%%%%%%%%%%%%%%%%%%
\usepackage[figure,table]{hypcap}% Correct a problem with hyperref
\urlstyle{rm} %so it doesn't use a typewriter font for url's.

% Hyperstuff
% \hypersetup{citecolor=webbrown,linkcolor=ChadDarkBlue,urlcolor=ChadDarkBlue}


%%%%%%%%%%%%%%%%%%%%%


% Fix title, sections, etc.

\let\LaTeXtitle\title
\renewcommand{\title}[1]{\LaTeXtitle{\color{ChadBlue}{\LARGE #1}}}
\renewcommand{\abstractname}{\color{ChadBlue}Abstract}
\renewcommand{\figurename}{\color{ChadBlue}Figure}
\renewcommand{\tablename}{\color{ChadBlue}Table}

\let\LaTeX@startsection\@startsection 
\renewcommand{\@startsection}[6]{\LaTeX@startsection%
{#1}{#2}{#3}{#4}{#5}{\color{ChadBlue}\raggedright #6}} 

%% % Fix periods at end of section numbers
%% \renewcommand \thesection {\@arabic\c@section.}
%% \renewcommand\thesubsection   {\thesection\@arabic\c@subsection}%.}
%% \renewcommand\thesubsubsection{\thesubsection \@arabic\c@subsubsection}%.}


% Add a period *only* after Section and *only* in title, not cross ref
% April 2017

\renewcommand{\@seccntformat}[1]{%
  \csname the#1\endcsname% Print sectional counter
  \ifnum\pdfstrcmp{#1}{section}=0 .\fi% If \section, print .
  \quad% Space between number and title
}




% Margin, Paragraph Setup

%\textheight=21.5cm \textwidth=14.95cm \topmargin=-5.5mm
%\oddsidemargin=8mm \evensidemargin=8mm
\textheight=22.5cm 
\textwidth=16.35cm 
\topmargin=0mm
\oddsidemargin=0mm 
\evensidemargin=0mm
%\setlength{\parindent}{0em} 
%\setlength{\parskip}{1.5ex plus0.5ex minus0.5ex}
%\setlength{\parsep}{0pt}

\setlength{\parskip}{4pt}

% This is how to control linespacing using setspace
% (\singlespacing or doublespacing just call this command)


%% Setting up page headers

\rhead[]{\thepage}
\lhead[\thepage]{}
\cfoot[]{} 
\renewcommand{\headrulewidth}{0pt}

\newcommand{\runningheads}[2]{
   \chead[\color{ChadGreen}{\uppercase {\footnotesize #1}}]  % Author page header
   {\color{ChadGreen}{\uppercase {\footnotesize #2}}}  % Short title
  }

% Make hyperlinks jump more accurately

\newcommand{\org@hypertarget}{}
\let\org@hypertarget\hypertarget
\renewcommand{\hypertarget}[2]{%
\Hy@raisedlink{\org@hypertarget{#1}{}}#2%
}  


% Spacing in Tables and Figures
%\renewcommand{\tabcolsep}{1pt}   % space between columns
\renewcommand{\arraystretch}{1.5} % space between rows
\addtolength{\textfloatsep}{0pt} % space between floats and text
\addtolength{\abovecaptionskip}{0pt} % space above caption
\addtolength{\belowcaptionskip}{.15in} % space below caption

\newcommand{\spc}[0]{\vspace{.1in}}
\newcommand{\fignote}[2]{\begin{center}\parbox[c]{#1}{\footnotesize #2} \end{center}}
\newcommand{\tabnote}[2]{\begin{center}\parbox[c]{#1}{\footnotesize #2} \end{center}}
\newcommand{\clr}[1]{{\color{ChadBlue} #1}}
\newcommand{\boxeq}[1]{\boxed{\hspace{.25in} #1 \hspace{.25in}}} 

\newcommand{\bb}[1]{\color{ChadBlue}{#1}}
\newcommand{\clrg}[1]{\color{ChadGreen}{#1}}
\newcommand{\gn}[1]{{\color{ChadGreen}{#1}}}
\newcommand{\gr}[1]{\tiny\textcolor{green4}{#1}}
\newcommand{\rd}[1]{\textcolor{ChadRed}{#1}}
\newcommand{\rr}[1]{\textcolor{red}{#1}}
\newcommand{\imp}[0]{$\Rightarrow \,$} % implies

%% Shortcut commands
\newcommand{\growth}[1]{\frac{\dot{#1}_t}{{#1}_t}}
\newcommand{\prtl}[2]{\frac{\partial #1}{\partial #2}}
\newcommand{\prtlsec}[2]{\frac{\partial^2 #1}{\partial #2^2}}
\newcommand{\commentno}[2]{\underset{\mbox{#2}}{#1}}
% \newcommand{\comment}[2]{\underbrace{#1}_{\mbox{{\small \color{ChadGreen}#2}}}}
\newcommand{\cn}[1]{\citet*{#1}}
\newcommand{\cnp}[1]{(\citealt{#1})}  % (Jones 2002)
\newcommand{\up}[0]{\uparrow \hspace{-.5ex}}  % \! is a negative thinspace
\newcommand{\down}[0]{\downarrow \hspace{-.5ex}}
\newcommand{\query}[1]{{\it \color{red}[??? #1]}}
\newcommand{\Ex}[1]{\mbox{ }\mathbb{E}}

% Use of Proof...
\newcommand{\Proof}[2]{{\hspace{-\parindent} {\color{ChadGreen}\bf Proof of Proposition}~\ref{#1}.}
{\color{ChadBlue} #2} \vspace{.1in}}
\newcommand{\Prob}[1]{\mbox{\textnormal{Pr}} \, [#1] }
\newcommand{\SubSubSection}[1]{\subsubsection{#1} \baselineskip=18pt}
% \newcommand{\problem}[2]{\hspace{.1in} {{\color{ChadBlue}{\bf #1:} #2}}}
\newcommand{\proptitle}[1]{\color{ChadBlue} \textnormal{(#1):}}

%\newcommand{\fignote}[2]{\begin{center}\parbox[c]{#1}{\footnotesize #2} \end{center}}
%\newcommand{\tabnote}[2]{\begin{center}\parbox[c]{#1}{\footnotesize #2} \end{center}}
% \newcommand{\definition}[1]{\mbox{\textnormal{Pr}} \, [#1] 

% \newcommand{\assume}[2]{{\bf{Assumption #1}} (#2)} 
% For graphs of networks
\usepackage{tikz}
\usetikzlibrary{arrows.meta,decorations.pathmorphing,backgrounds,positioning,fit,graphs.standard}%finish for graphs of networks

\newtheorem{theorem}{\color{ChadGreen} Theorem}
\newtheorem{proposition}{\color{ChadGreen} Proposition}
\newtheorem{problem}{\color{ChadGreen} Problem}
\newtheorem{definition}{\color{ChadBlue} Definition} 
\newtheorem{corollary}{\color{ChadGreen} Corollary} 
\newtheorem{lemma}{\color{ChadGreen} Lemma}
\newtheorem{assumption}{\color{ChadBlue} Assumption}

%\newtheorem{prob}{\color{ChadBlue} Problem}\newcommand{\assume}[2]{{\bf{Assumption #1}} (#2)} 
%\newtheorem{exercise}{\color{ChadBlue} Problem}\newcommand{\assume}[2]{{\bf{Assumption #1}} (#2)} 

\newcommand{\pfrac}[2]{\left( \frac{#1}{#2} \right)}  % frac with large paren

\def \AggS {X}
\def \sign {\text{sign}}

\DeclareMathOperator*{\argmax}{arg\,max}
\DeclareMathOperator*{\argmin}{arg\,min}

\makeatother

\usepackage{babel}
\addto\captionsamerican{\renewcommand{\corollaryname}{Corollary}}
\addto\captionsenglish{\renewcommand{\corollaryname}{Corollary}}
\providecommand{\corollaryname}{Corollary}

\interfootnotelinepenalty=10000