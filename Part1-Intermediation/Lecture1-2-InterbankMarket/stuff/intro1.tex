
A growing empirical literature documents that certain assets command
substantial convenience yields---premia that cannot be explained
by their cash flows alone. The pattern appears across short-term assets
such as U.S. Treasuries \citep{Krishnamurthy2012}, cash-like instruments
\citep{Nagel2016}, and synthetic dollars in international markets
\citep{JiangKrishnamurthyLustig2021,EngelWu2023}. Convenience yields
vary substantially over time and across assets classes, with important
implications for monetary policy transmission, fiscal capacity, and
international capital flows. While convenience yields have become
central to understanding asset prices and macroeconomic policy, their
theoretical foundations remain incomplete. This paper develops a tractable microfoundation for convenience yields arising from trading frictions in over-the-counter (OTC) financial markets and incorporates it into canonical portfolio theory.

%by integrating portfolio theory
%with the theory of over-the-counter (OTC) markets.

In the theory, investors are subject to payoff risk---stemming from fundamental variations in returns---and liquidity risk.
%convenience yields arise endogenously from settlement
%frictions in financial markets. In addition to payoff risk,
%In our model, 
%Financial
%positions expose investors to liquidity risk when
Liquidity risk emerges because cashflows are unpredictable, and financial imbalances must be settled in an OTC market. Examples of financial
positions exposing investors to liquidity risk
include deposit withdrawals for banks, margin calls for hedge funds,
or claim payouts for insurance companies. To meet these obligations,
investors hold buffers of liquid assets that can be deployed for settlement.
When liquidity needs exceed available buffers, investors must borrow
settlement instruments in a frictional OTC market. 
The interaction
between portfolio-induced settlement needs and OTC market frictions
generates an endogenous convenience yield that depresses the returns
on liquid assets. Moreover, to the extent that settlement needs   correlate with
asset returns, this creates an additional a liquidity-risk
premium in excess of the conventional risk premium. Thus, convenience yields
reflect both  a first-order effect  and the covariance between
settlement needs and asset payoffs. Crucially, because these yields
depend on portfolio choices and OTC market conditions, they are not
invariant to policy and vary with market tightness and trading efficiency
of the OTC market.

Our analysis yields a tractable derivation of convenience yields that can be easily embedded in standard portfolio problems.
% %makes three main contributions. First, for portfolio
% theory, we demonstrate how settlements frictions show up as a liquidity
% return function that depends on both individual investors' settlement
% needs and market tightness for the settlement instrument. This allows
% OTC frictions to be tractably embedded in standard portfolio problems,
% with convenience yields emerging naturally from optimization. Second,
% for the OTC literature, we model a sequential OTC market and characterize
% the rates and trading probabilities along all trades for a general
% class of matching functions. We furthermore provide explicit formulas
% for Leontief and Cobb-Douglas matching functions. Third, we show that
% convenience yields interact with risk premia and respond endogenously
% to changes in market structure, matching efficiency, and policy interventions.
% Our analytical formulas enable comparative statics that link OTC market
% parameters to observable trading patterns and convenience yields.
%To integrate portfolio theory with an OTC market for settlement instruments,
To do so, we build on the sequential OTC framework of \citet*{AL15},   with two key innovations.
First, instead of taking the settlement positions as
given, we endogenize them by modeling the portfolio choice  of investors.
% through portfolio choice: agents choose
% portfolios considering both asset return risk and potential settlement
% needs, then trade in the OTC market when settlement shocks realize.
Second, we model portfolio managers as large institutions who delegate
settlement trades to many small traders, following \citet{Shi1997}
and the OTC model of \citet*{AEW15}. Taking limits as trader size
vanishes yields closed-form characterizations of the entire equilibrium
path of trades and rates.
%---a departure from the finite-trader framework
%in Afonso-Lagos that requires numerical simulations. 

% Our version of the \citet*{AL15} market is essential for embedding
% settlement frictions into portfolio theory and deriving comparative
% statics. In particular, 

Settlement frictions manifest
as a piecewise linear liquidity yield function $\chi\left(s\right)$
that maps settlement positions $s$ for each asset into an additional portfolio return.
%Each asset creates a specific exposure on
%$s$, inducing a rich interaction between portfolio choice and OTC
%trading. 
This function is kinked at zero---reflecting the asymmetry
between borrowing costs and lending returns characteristic of the OTC market.
Its slopes are, exactly, the expected lending return for investors
with surplus and the borrowing costs for those with deficits that
stem from the OTC market. These convenience-yield coefficients depend
on only three objects: an endogenous market tightness (the ratio of
aggregate deficits to surpluses), a bargaining parameter, and the
matching technology. The asymmetry resulting from the kink induces risk-averse behavior even under risk-neutral preferences and implies that assets that generate volatile settlement needs command higher
convenience yields. Moreover, when settlement
needs correlate with asset returns---as when margin calls intensify
in downturns---an additional liquidity risk premium emerges. Thus
convenience yields reflect both pure OTC frictions and the interaction
between liquidity risk and return risk. 

Our analysis reveals several important properties linking OTC markets
to convenience yields. First, the choice of matching technology fundamentally
alters market dynamics: under Cobb-Douglas matching, one side of the
market can vanish in finite time, causing trading to cease and convenience
yields to reach their bounds; under Leontief matching, the short side
always trades, preventing complete market freeze. This distinction
helps explain a puzzling observation in interbank markets where minimal
trading persists even when rates hit the floor \citep[e.g., see][]{LopesSalidosVissingJorgenson,AfonsoGiannoneLaSpadaWilliams,LagosNavarro2023}.\footnote{We also show that only certain matching functions, those with
infinite-rate of decay, can generate near-zero convenience yields
with positive trading volumes.} Second, we establish that convenience yields satisfy time dilation (the passage of time is mathematically equivalent
to reduced matching efficiency) and symmetry (reversing market tightness
and swapping bargaining powers yields identical surplus extraction).
Third, our comparative statics reveal sharp predictions: market tightness
unambiguously increases convenience yields by steepening the liquidity
yield function, while matching efficiency has non-monotonic effects. The intuition for the latter is that higher matching efficiency benefits the short side of the market, raising convenience yields
when liquidity is scarce but potentially lowering them when abundant. 
Furthermore, our closed-form solutions can  aid identification of typically unobservable OTC market
parameters and shocks from observable data for trading volumes and spreads. 

% Monotonicity properties
% are crucial for identification. Our comparative statics analysis reveals
% that convenience yields and average OTC rates increase monotonically
% with the market tightness, they may be non-monotonic in matching efficiency.
% In turn, the OTC market's volume is monotonic in efficiency. This
% suggests a robust identification approach: identifying market tightness
% from convenience yields or average OTC rates while using trading volume
% to identify the matching efficiency. For instance, in the federal
% funds market, this approach allows to links observable spreads, volumes,
% and emergency borrowing to the underlying market structure.

% The comparative statics results generate sharp empirical predictions
% that enable identification of typically unobservable OTC market parameters
% and settlement needs. For example, rate dispersion within the trading
% day identifies market tightness---the dispersion of rates increases
% with the distance from balanced OTC markets. In turn, the relative
% trading volume pins down matching efficiency. For instance, in the
% federal funds market, our model links observable spreads, volumes,
% and borrowing patterns to underlying liquidity demand shocks studied
% in recent work. This provides a structural interpretation of convenience
% yield variation and a new approach to quantifying settlement frictions.

%In addition to characterizing convenience yields, 
Finally, we show how the interaction between portfolio choice with settlement frictions generate a pecuniary externality that induces an inefficient investment in liquid assets.
Individual investors, when choosing portfolios, do not internalize
how their settlement needs affect aggregate market tightness and thus
the liquidity yields faced by all market participants. This creates
a wedge between private and social returns to holding liquid assets.
We characterize conditions under which competitive equilibria feature
over- or under-investment in liquidity. We show that with risk neutrality, investment in liquid assets is insufficient when the marginal impact of tightness on borrowing costs exceeds its impact on lending returns. This occurs in turn when aggregate surplus exceeds deficits. With risk
aversion, the inefficiency is amplified as investors fail to account
for how their portfolios affect others' liquidity risk. Strikingly,
under balanced markets (equal settlement instrument deficits as surpluses)
with symmetric shocks and Cobb-Douglas matching, the competitive equilibrium
is constrained efficient under risk neutrality---but exhibits under-provision
of liquidity under risk aversion. 
% These results differ fundamentally
% from existing OTC externalities based on composition effects \citet{Uslu2019}
% or search intensity \citet{WongZhang2023}, as our externality operates
% through the portfolio-induced distribution of settlement needs.

The findings have  implications for liquidity regulation , suggesting that optimal policy depends critically on  market conditions.

%the OTC market structure here can be used to study those questions.
% These studies consider a specific version of the matching technology
% presented here. This paper generalizes the convenience yield function
% adopted in those approaches: (i) deriving the convenience yield for
% arbitrary matching technologies, revealing when convenience yields
% can vanish (Cobb-Douglas) versus persist (Leontief); (ii) providing
% the closed-form that enable empirical identification from OTC market
% or convenience yield data. Regarding the portfolio aspects, relative
% to that work, here we (i) allow for heterogeneity in settlement risks,
% making the framework applicable beyond banking applications; and (ii)
% uncovering the pecuniary externality with implications for liquidity
% regulation. Thus, while our earlier work demonstrated the usefulness
% of liquidity yield functions in specific contexts, this paper provides
% the general partial-equilibrium foundation that can be embedded in
% various general-equilibrium applications. 

\paragraph{Literature Review. }

Our paper is related to a large literature on portfolio choice and
asset pricing with liquidity frictions, such as portfolio constraints
or transaction costs. Important examples in this literature include
\citet{Constantinides1986,Basak1998,Vayanos1999,He2013} \citet{constanidesduffie1996,KruegerLustig2010},
Acharya, Luttmer...milbrandt, \citet{Holmstrom2001}, \citep[see][]{lagos2010asset}.
Typically, these studies model liquidity frictions as exogenous. Our
goal in this paper is to develop a tractable framework where liquidity
premia emerges endogenously and explore the implications for asset
portfolios.

Following \citet*{Duffie2005}, a burgeoning literature on OTC markets
has studied environments where assets are traded in presence of search
frictions.\footnote{This literature has ran in parallel to the money search literature,
pioneered by Kiyotaki and Wright. See, \citet{williamson_wright_2010_models,Lagos2017}
for recent surveys.} This literature has identified how features of the trading environment
such as the speed of transactions and the heterogeneity in the motives
for trade affect trading volumes and impact liquidity premia. While
the literature began with strong restrictions on portfolio holdings,
namely binary holdings, work by \citet{Lagos2009} allow for arbitrary
portfolio holdings bringing this literature closer to standard portfolio
theory. \citet{HuggonierLesterWeill22} consider heterogeneity in
asset selling speeds and \citet{Uslu2019} introduced risk-averse
behavior into this class of models. In these models, trading speeds
affect asset values because time is discounted, a feature that depresses
the option value of selling assets when gains from trade emerge. \citet{SilvaPassadoreKargar}
take a step further by studying portfolio problems that explicitly
take into account trading times when agents want to modify their portfolios.\footnote{In our model, time discounting plays no role, as trading occurs within
a single period although the sequence of trades matters because it
affects the terms of trade when two investors match.} In the language of Lester et al., these papers study a semi-centralized
setup while we study a purely decentralized setup. .

The OTC block of our model is very similar to \citet*{Afonso2015}
where agents are restricted to have $\left\{ -1,1\right\} $ settlement
positions. While the matching structure is different, some of the
formulas here resemble those found in \citet{Afonso2015} which characterize
the volumes and terms of trade. Relative to that work, a contribution
of our work is to show, that by working with a large number of traders,
as in \citet{Shi1997}, and \citet*{AEW15}, the OTC block can be tractably
incorporated to portfolio theory. In addition we derive comparative
statics with respect to the dispersion of rates, outside funding,
and the slopes of the liquidity yields. These features allow us to
link moments regarding the dispersion in rates and outside funding
to liquidity premia.

Our normative analysis is related to a broad literature on the welfare
properties of competitive equilibrium with financial frictions, in
particular, those studies analyzing the efficiency of risk-taking
decisions. One branch of the literature more closely related focuses
on over or under-investment in liquid assets (\citealp{Jacklin1987,Bhattacharya1987,Farhi2009,Yared2013,Geanakoplos2017}).
These studies consider a Walrasian interbank market where the risk-free
rate affects the degree of enforcement and risk sharing. In contrast,
we consider a setting with an OTC market where a pecuniary externality
emerges from congestion in the interbank market. The externality we
study is related to other congestion externalities in the search and
matching literature. In particular, \cite{Uslu2019} identify an externality where
fast investors able to capture a private transaction surplus larger
than their contribution to surplus creation \footnote{In contrast, in a model with exogenous search effort, \cite{afonso2015over}
finds that the equilibrium implements the efficient reallocation of
reserves.} In addition, \cite{WongZhang2023} show the degree of search and intermediation
is inefficient in the OTC market where search intensity is endogenous.
Different from these contributions, our approach compares the portfolio
choices of individual investors vis a vis a social planner that takes
as given the financial arrangements in the OTC market. In this respect,
our analysis is closer to \cite{Arseneau2017} who study inefficient liquidity
provision in a model where firms issue long-term bonds that are retraded
by investors in an imperfectly liquid secondary market.\footnote{A different strand of the literature analyzes pecuniary externalities
that can result in excessive leverage (see e.g. \citealp{caballero2001international,Lorenzoni2008,bianchi2011overborrowing,davila2018pecuniary,amador2024bank}). }


Finally, we emphasize that while our earlier work (\citealp{BB17}) employs the convenience yield function derived here to analyze the transmission and implementation of monetary policy through the banking system, the present paper provides its theoretical foundations.\footnote{An additional contribution relative to \citet{BB17} is the analysis of efficiency of portfolio choices.} Other applications include \citet{Arce2019} and \citet{Bigio2019}, who study optimal reserve policy; \citet{Bianchi2020}, who examine exchange rate determination in an international context; and \citet{bittner2025assortative}, who apply the framework to the German interbank market with assortative matching.\footnote{In \citet{Piazzesi2019} and \citet{Lenel2019}, settlement risks are used to explain the short-term rate puzzle and the determinacy of interest rate rules.}

