\documentclass[10pt]{beamer}
%%%%%%%%%%%%%%%%%%%%%%%%%%%%%%%%%%%%%%%%%%%%%%%%%%%%%%%%%%%%%%%%%%%%%%%%%%%%%%%%%%%%%%%%%%%%%%%%%%%%%%%%%%%%%%%%%%%%%%%%%%%%%%%%%%%%%%%%%%%%%%%%%%%%%%%%%%%%%%%%%%%%%%%%%%%%%%%%%%%%%%%%%%%%%%%%%%%%%%%%%%%%%%%%%%%%%%%%%%%%%%%%%%%%%%%%%%%%%%%%%%%%%%%%%%%%
\usepackage{amsfonts}
\usepackage{graphicx}
\usepackage{booktabs}
\usepackage{hyperref}
\usepackage{epsfig}
\usepackage{color}
\usepackage{amsmath,amssymb}
\usepackage[latin1]{inputenc}
\usepackage{psfrag}
\usepackage{epstopdf}
\usepackage[disable]{todonotes}
\usepackage[disable]{todonotes}
\usepackage{setspace}
\setcounter{MaxMatrixCols}{10}
\geometry{left=0.65cm,right=0.8cm}
\renewcommand{\sfdefault}{cmss}
\usefonttheme{serif}
\def\widehrulefill{\leaders\hrule height 0.35pt\hfill}
\def\drawhline{\hrule height 0.6pt width 333pt}
\def\no{\noindent}
\def\bul{\color{black}{$\bullet$}}
\def\cir{\color{black}{$\circ$}}
\def\br{\color{black}{-}}
\def\wbul{\color{white}{$\bullet$}}
\def\wcir{\color{red}{$\circ$}}
\def\wx{\color{white}{x}}
\def\bx{\color{blue}}
\def\rx{\color{red} }
\def\nor{\normalsize}
\def\bsk{\bigskip}
\def\msk{\medskip}
\def\ssk{\smallskip}
\setbeamertemplate{frametitle}
{
\bigskip
\definecolor{ltblack}{gray}{0.1}
\color{ltblack}{\large\textbf{{\insertframetitle} }}
}
\definecolor{defblue}{rgb}{.2,.2,.7}
\definecolor{purple}{rgb}{0.9,0.0,0.0}
\setbeamertemplate{navigation symbols}{}
\setbeamertemplate{footline}{
  \hbox{
  \begin{beamercolorbox}[wd=.49999\paperwidth,ht=2.25ex,dp=1ex,center]{}
  \insertshortauthor
  \quad
  \end{beamercolorbox}
  \begin{beamercolorbox}[wd=.49999\paperwidth,ht=2.25ex,dp=1ex,left]{}
    \quad
    \insertshorttitle
    \hfill \quad\quad\quad\quad\quad
    \insertframenumber{} / \inserttotalframenumber
  \end{beamercolorbox}}
  \vskip0pt
  }
\setbeamertemplate{itemize items}[ball]
\setbeamertemplate{itemize subitem}{\color{black} $\circ$}
\setbeamertemplate{itemize subsubitem}{\color{black} -}
\setbeamerfont{itemize item}{size=\small}
\setbeamerfont{itemize subitem}{size=\small}
\setbeamerfont{itemize subsubitem}{size=\small}
\setbeamerfont{itemize subsubitem}{size=\small}
\newcommand{\fs}{\mathcal{S}}
\newcommand{\RR}{\mathcal{R}}
\DeclareMathSymbol{\R}{\mathbin}{AMSb}{"52}
\newtheorem{defn}[theorem]{Definition}
\newtheorem{claim}[theorem]{Claim}
\newtheorem{assumption}[theorem]{Assumption}
\newtheorem{proposition}[theorem]{Proposition}
\fontsize{1pt}{7.2}\selectfont
\setbeamertemplate{footline}{\raisebox{5pt}{\makebox[\paperwidth]{\hfill\makebox[10pt]{\scriptsize\insertframenumber}}}}
\setbeamertemplate{itemize items}[ball]
\graphicspath{{../New_javi_aug/Images_javi/}{../empirical/}}
\newenvironment{wideitemize}{\itemize\addtolength{\itemsep}{30pt}}{\enditemize}
\newcommand\unnumbered{\setbeamertemplate{footline}}
\setbeamercolor{page number in head/foot}{fg=white}
\setbeamertemplate{footline}[page number]{}
%\input{tcilatex}
\begin{document}

\title[Banks and Liquidity]{
\textcolor[rgb]{0.50,0.00,0.25}{\bf Banks,
Liquidity Management \\  and Monetary Policy}}
\author[Bianchi and Bigio]{Javier Bianchi \and \quad \quad \quad Saki Bigio}
\institute[UMP]{Minneapolis Fed \& NBER \qquad \qquad \hspace{10pt} UCLA \&NBER }
\date{}


\vspace{2cm}

\newcommand*{\Scale}[2][4]{\scalebox{#1}{$#2$}}%
\newcommand*{\Resize}[2]{\resizebox{#1}{!}{$#2$}}%


\frame{\titlepage}

\section{Motivation}

  \linespread{1.5}
\section{Introduction \leaders\hrule height 0.35pt}

% 20 min
\graphicspath{{../Images/}}
\begin{frame}
\frametitle{Motivation \widehrulefill}

\begin{itemize}
  \setlength{\itemsep}{15pt}

\item Financial crisis raised unprecedented challenges for central banks: \medskip

\begin{itemize}
  \setlength{\itemsep}{3.5pt}
\item Worst recession since Great Depression
\item Persistently low inflation
\end{itemize}

%\pause

\item Despite aggressive monetary policy

\begin{itemize}
\item Weak economic recovery
\item Depressed bank lending
%hyperlink{F_OMO}{\beamergotobutton{Data}} %, close to zero int. rates, etc
\end{itemize}

\pause

\item Center of debate: banks' reaction to monetary stimulus

\begin{itemize}
\item Why are banks \alert{not lending}? % \hyperlink{F_reserves}{
%\beamergotobutton{Data}} \bigskip
% \hypertarget{DataSlide}{}
\end{itemize}
%\pause
\item Unfortunate divorce between Money and Banking
\end{itemize}
\end{frame}

\begin{frame}
\frametitle{Bank Lending   \widehrulefill}
\begin{figure}[h!]
\begin{centering}
\includegraphics[width=11cm,height=8cm]{Plot_loans}
%\caption{Loans and Leases in Bank Credit}
%\label{fig:figure1}
\end{centering}
\end{figure}
\end{frame}

\begin{frame}
\frametitle{Bank Lending and Deposits   \widehrulefill}
\begin{figure}[h!]
\begin{centering}
\includegraphics[width=11cm,height=8cm]{plot}
%\label{fig:figure1}
\end{centering}
\end{figure}
\end{frame}

\begin{frame}
\frametitle{Excess Reserves\widehrulefill}
\begin{figure}[tbp]
\includegraphics[width=180pt, angle=270,scale=1.20]{dataimages/F_LMMP_excessreserves}
%\caption{Commercial and Industrial Loans}
\end{figure}

\end{frame}
%
%%%%%%%%%%%%%%%%%%%%%%%%%%%%%%%%%%%%%%%%%%%%%%%%

\begin{frame}
\frametitle{This Paper \widehrulefill}
\begin{itemize}
 % \renewcommand{\labelitemi}{\scriptsize$\blacksquare$}
\item[$\bigstar$] A micro-founded dynamic quantitative model of banks' liquidity management and monetary policy
\end{itemize}
    \medskip
    \pause
\begin{itemize}
\item \alert<2>{Classic Liquidity Management }
\begin{itemize}
\item  (+) Profit on Loans
\begin{itemize}
\item Spread between loans and deposits
\end{itemize}
\item  (-) Illiquidity Risk
\begin{itemize}
\item After deposits transfers, bank may be short of reserves/liquid assets
\end{itemize}
\end{itemize}
\pause
\item [$\bigstar$] GE model with endogenous liquidity premium $\&$ money multiplier
\pause
\medskip
\item[$\bigstar$] Credit channel of monetary policy
\end{itemize}
\end{frame}



\begin{frame}
\frametitle{Model Overview  \widehrulefill}

\begin{enumerate}
\setlength{\itemsep}{8pt}
\item \alert<1>{Bank individual decision problem:}

\begin{itemize}
\item Loan issuances, liquid assets, deposit creation, retain earnings

\item Subject to capital requirements, liquidity requirements

\item [$\bigstar$] Demand for excess reserves
\end{itemize}

\pause
\bigskip

\item \alert<2>{Bank industry dynamics:}

\begin{itemize}
\item Return on loans, interbank-market rate, price level determined in
equilibrium

\item Stochastic steady state and transitional dynamics
\end{itemize}
\pause
\item \alert<3>{Laboratory for policy analysis:} changes in reserve
requirements, IOER, capital requirements, discount window rates,
conventional/unconventional policies
\pause
\item [$\bigstar$]  \alert<4>{Tractability}
\end{enumerate}
\end{frame}

%%%%%%%%%%%%%%%%%%%%%%%%%%%%%%%%%%%%%%%%%%%%%%%%%%%

\begin{frame}
\frametitle{Application \widehrulefill}

\begin{itemize}
\item Why are banks \alert{not lending} and \alert{``stockpiling"} liquid
assets?

\bigskip \pause

\item Five Hypothesis

\begin{enumerate}
 \setlength{\itemsep}{7pt}

\item Equity Losses (Stiglitz) \pause

\item Capital Requirements (Krishnamurthy, Suarez) \pause

\item \alert<3>{Precautionary Holdings} (Lucas, Gorton) \pause

\item \alert<3>{Weak Loan Demand} (Cochrane, Sims) \pause

\item \alert<4>{Interest on Excess Reserves} (Feldstein, Hall)

% \item Open Market Operations (Keister) \pause
\end{enumerate}

\end{itemize}
\end{frame}




%%%%%%%%%%%%%%%%%%%%%%%%%%%%%%%%%%%%%%%%%%%%%%%%%%%%%%
\linespread{1.5}
\begin{frame}
\frametitle{Literature Review  \widehrulefill}

\begin{itemize}
%\item Call for studying banks in transmission of MP in Macro:
%\begin{itemize}
%\item Woodford (2010, JEP), Mishkin (2012, JEP), Greenwood \& Stiglitz (2003)
%\end{itemize}
\item \textbf{Reserve Management:} \alert<2>{Poole (JF,1968)},Bolton et al. (2012),
Saunders et al. (2011), Afonso \& Lagos (2015)
\item \textbf{Classic models of Banking:} Diamond \& Dybvig (1983), Allen \& Gale (1998), Holmstrom \&
Tirole (1997,1998)
\item \textbf{Banking in Macro:}  Gertler \& Karadi(2009), Gertler \& Kiyotaki (2011,2012), Curdia \&
Woodford(2009), Corbae \& D'erasmo (2013,2014) %, Kiyotaki \& Moore (2008)
\item \textbf{Payments:} Freeman(AER,1996), Cavalcanti et al. (1998), Piazzesi and
Schneider (2015)
\item \textbf{Money \& credit:} , Wright et al. (2014),  Brunnermeier \& Sannikov (2013), Williamson (2012,2016), Kiyotaki \& Moore (2012)
\item \textbf{Excess Reserves:}  Armenter \& Lester (2015), Ennis (2014)
%\end{itemize}

%\item {\tiny Empirical Work }
%
%\begin{itemize}
%\item {\tiny Kashyap \& Stein(1998), Krishnamurthy \& Vissing-Jorgenson (JPE
%2012a,2012b), }
%\end{itemize}


%\item \alert<1>{Contribution:} Classic banks' liquidity management in a
%micro-founded dynamic quantitative model
\end{itemize}
\end{frame}
\linespread{1.5}
%%%%%%%%%%%%%%%%%%%%%%%%%%%%%%%%%%%%%%%%%%%%%%%%%%%%%

%%%%%%%%%%%%%%%%%%%%%%%%%%%%%%%%%%%%%%%%%%%%%%%%%%%%%

%\begin{frame}
%\frametitle{Agenda  \widehrulefill}
%\begin{enumerate}
%\item Banks' Liquitidy Management Problem
%\pause
%\bigskip
%\item Stationary Stochastic Steady State
%\begin{itemize}
%\item Static Demand for Loans
%
%\item We won't talk about microfoundation
%\end{itemize}
%\pause
%\bigskip
%\item Transitional Dynamics Experiments
%\begin{itemize}
%\item Understand effects of shocks
%
%\item Explore MP
%\end{itemize}
%
%\pause
%\bigskip
%\item Estimation of model shocks to evaluate four hypotheses (in progress)
%\end{enumerate}
%\end{frame}

%%%%%%%%%%%%%%%%%%%%%%%%%%%%%%%%%%%%%%%%%%%%%%%%

\section{Model}

\begin{frame}
\frametitle{Model - Environment \widehrulefill}

\begin{itemize}
\item \textbf{Time:} t=1,2,3,....

\pause

\begin{itemize}
\item \textbf{Two stages:} s=l,b

\item Lending stage (l) and balancing stage (b)
\end{itemize}

\pause\bigskip

\item Continuum of banks with idiosyncratic liquidity shocks %$z\in \left[ 0,1\right] $

\pause\bigskip

\item \textbf{Utility function: }Concave utility $U$ over dividends $c_{t}$,  $u(c)=%
\frac{c^{1-\sigma }}{{1-\sigma }}$
\end{itemize}
\end{frame}

\subsection{Bank Liabilities}

\begin{frame}
\frametitle{Banks' Balance Sheet  \widehrulefill}

\begin{itemize}
\item Liabilities:
\begin{itemize}
\item $d_{t}$ demand deposits %(\emph{numeraire})
\end{itemize}
\item Assets:
\begin{itemize}
\item $m_{t}$ liquid assets (central bank reserves, T-bills) %\begin{itemize}
%\item $p_{t}$ price of reserves in terms of goods
%\end{itemize}
\item $b_{t}$ loans
\end{itemize}
\item Interbank market loans $f_t$ and discount-window loans $w_t$ %\item Let equity by $E\equiv
\item All assets/liabilities denominated in nominal terms
\end{itemize}
\end{frame}

\begin{frame}
\frametitle{Banks' Problem (Lending Stage)\widehrulefill}
\begin{itemize}
\item Budget constraint
%\begingroup\makeat `letter\def\f@size{10.47}\check@mathfonts
\begin{align*}
P_{t}c_{t}^{j}+\tilde{b}_{t+1}^{j}+\tilde{m}_{t+1}^{j}-\tilde{d}^{j}
_{t+1}&=b^{j}_{t}(1+i_{t}^{b})+m_{t}^{j}(1+i_{t}^{ior})-d_{t}^{j}(1+i_{t}^{d})-\\
& \left( 1+\bar{\imath}_{t}^{f}\right) f_{t}^{^{j}}-\left( 1+i_{t}^{dw}\right)
w_{t}^{j}-P_{t}T_{t}
\end{align*}


%\begin{align*}
%q_t B_{t+1} + P_t {C}_t + \tilde{m}_{t+1} + \frac{\tilde{d}_{t+1}}{1+i_t^D} &=
%B_t + M_t(1+i_{t}^{ior}) - D_t \\
% & + F_t - W_t +T_t % ADD DISCOUNT WINDOW LOANS HERE AND T
% % ADD R^FF HERE AND TAKE IT OUT FROM THERE
%\end{align*}  %$R_{t} determined at t-1$
%\normalsize
\item $P_t$ price of goods in terms of reserves
\pause
\item Capital requirement constraint
\begin{align}
\tilde{d}^{j}& \leq \kappa \left( \tilde{b}^{j}+\tilde{m}^{j}-\ \tilde{d}^{j}\right)
\end{align}
%\pause
%\item Liquidity coverage ratio (not today)
%\begin{equation*}
%\tilde{m}_{t+1} \ge \rho^{LCR} \tilde{b}_{t+1}
%\end{equation*}
\end{itemize}
\end{frame}

\linespread{0.8}
\begin{frame}
\frametitle{Banks' Problem (Balancing Stage)\widehrulefill}
\begin{itemize}
\item Idiosyncratic withdrawal shock $\omega \in (-1, \infty], \omega \sim  F_{t}\left(\omega
\right), \quad \mathbb{E}\left( \omega \right) =0 $  %to deposits
%(payments/confidence)
\smallskip

\item Borrow in interbank market $f$ to satisfy res. req. (or $m_{t+1}\geq 0$)
\medskip

%\begin{align*}
%d_{t+1}^{j}& =\tilde{d}_{t+1}^{j}+\omega _{t}^{j}\tilde{d}_{t+1}^{j}, \\
%m_{t+1}^{j}& =\tilde{m}_{t+1}^{j}+\omega _{t}^{j}\tilde{d}_{t+1}^{j}\left(
%\frac{1+i_{t+1}^{d}}{1+i_{t+1}^{ior}}\right) \ +f_{t+1}^{j}\ +w_{t+1}^{j},
%\end{align*}%

\begin{align*}
d_{t+1}^{j} &=\tilde{d}^{j}_{t+1} + \alert<2>{\omega _{t}}\tilde{d}_{t+1} ^{j}\\
m_{t+1}^{j}& =\tilde{m}_{t+1}^{j}+\alert<2>{\omega _{t}^{j}}\tilde{d}_{t+1}^{j}\left(
\frac{1+i_{t+1}^{d}}{1+i_{t+1}^{ior}}\right) \ +f_{t+1}^{j}\ +w_{t+1}^{j},\\
%\end{align*}%
m_{t+1}^{j} &\geq \alert<2>\rho d_{t+1}^{j} \\
\\
s^{j} &\equiv  \underbrace{\tilde{m}^{j}_{t+1}-\alert<2>{\omega} \tilde{d}^{j}_{t+1}}_{\text{Reserves left}}- \underbrace{\rho \tilde{d}^{j}_{t+1}(1-\alert<2>{\omega} )}_{\text{Deposits left}}
\end{align*}
\end{itemize}
\end{frame}


\begin{frame}
\frametitle{Interbank Market}
\begin{itemize}
  \setlength{\itemsep}{14pt}
\item ``Dollar to dollar"  matching (OTC markets of Atkeson et al. 2014)
\item N rounds of matching
\item Matching function for dollars in surplus ($S^+$) and deficit ($S^-$)
%\begin{equation*}
% \gamma_{n}^{-}= \gamma_0 \min \left( 1,\frac{S_{n}^{+}}{S_{n}^{-}}\right) \text{ and } %
% \gamma_{n} ^{+}= \gamma_0 \min \left( 1,\frac{S_{n}^{-}}{S_{n}^{+}}\right) .
%\end{equation*}
%\item Value for unmatched FED funds:%
%\[V_{n}^{+}=\gamma _{n+1}^{+}r_{n+1}^{I}+\left( 1-\gamma _{n+1}^{+}\right)
% V_{n+1}^{+}\]
%and
%\[
%V_{n}^{-}=\gamma _{n+1}^{-}r_{n+1}^{I}+\left( 1-\gamma _{n+1}^{-}\right)
%   V_{n+1}^{-} .
%\]
\item Terminal outside options are discount window rates and interest on reserves %$V_{N+1}^{+}=i^{ior};V_{N+1}^{-}=i^{dw}$.
\item Nash Bargaining
\item See paper for analytical results for Fed funds rate and matching probabilities
%\[
%r_{n}^{I}=\arg \max \left( V_{n}^{-}-r_{n}^{I}\right) ^{\eta }\left(
%r_{n}^{I}-V_{n}^{+}\right) ^{1-\eta }.
%\]

%\item Let  $M^{-}(M^{+})$ be the measures of dollars in deficit  (surplus)
%\item Endogenous Probability of ``per-dollar'' match:
%
%\item Bargaining Problem of dollar match:%
%\begin{equation*}
%  \max_{\alert<2>{r^{I}}}\left( r_{t}^{dw}-\alert<2>{%
%r^{I}}\right) ^{\xi }\left( \alert<2>{r^{I}}\right)^{1-\xi }
%\end{equation*}
%\pause
%\item Isomorphic to exogenous wedge in interbank market
\end{itemize}
\end{frame}


%\item Frictions in interbank market
%\begin{equation*}
%\chi ^{L}=\left\{
%\begin{array}{lll}
% 0 \text{ } \hspace{0.2cm} \text{if \ }L \le 0 \quad \\%\chi_{b}>1
%\chi_{b}\text{ }\text{if \ }L>0 \quad \chi_{b}>0
%\end{array}%
%\right.
%\end{equation*}
%\pause
%\item Optimal interbank market loan %accommodates deficit/lends surplus. Fed absorbs the average surplus/deficit
%\begin{equation*}
%L_{t+1}^{I,\ast }=\left\{
%\begin{array}{ll}
%-s(1+\chi _{t}^{L})\text{ }  \text{if \ }s<0 \\
%%\lbrack 0,s]\text{ } \quad \qquad \text{if \ }0 \geq 0, R^{FF}<I^{ior} \\  THIS CANNOT HAPPEN IN EQUILIBRIUM
%\lbrack 0,s]\text{ } \quad \qquad \text{if \ }s\geq 0, R^{FF}=I^{ior} \\ % if excess reserves
% s\text{ } \quad  \qquad  \qquad\qquad \text{if \ }s\geq 0, R^{FF}>I^{ior} \\  % if deficit on reserves
%\end{array}%
%\right.
%\end{equation*}
%\end{itemize}
%\end{frame}

%\begin{frame}
%Tightness by end of the trading round:
%\begin{equation*}
%\bar{\theta}_{t}=\left( \theta _{t}-1\right) \exp \left( \lambda \right) +1
%\end{equation*}
%
%\begin{proposition}[Interbank Market Trades]
%\label{Prop:InterbankMarketTrades} Define $\bar{\theta}_{t}$ as the market
%tightness at the end of a trading session.
%
%\textbf{Case 1:} If $\mu _{t}=1$. Then, $\psi _{t}^{+}=\psi _{t}^{-}=1-\exp
%\left( -\lambda \right) $. The average interbank-market rate is:
%\begin{equation*}
%\bar{\imath}_{t}^{f}=\frac{1+\left( 1-\exp \left( -\lambda \right) \right)
%\eta }{\left( 1-\exp \left( -\lambda \right) \right) }i^{ior}+i^{dw}\left(
%1-\eta \right) .
%\end{equation*}
%
%\textbf{Case 2:} If $\mu _{t}>1$. Then, $\psi _{t}^{+}=?,$ $\psi
%_{t}^{-}=1-\exp \left( -\lambda \right) $. \ The average interbank market is
%$\bar{\imath}_{t}^{f}=?$.
%
%\textbf{Case 3:}\textit{\ }If $\mu _{t}<1$. Then, $\psi _{t}^{+}=1-\exp
%\left( -\lambda \right) ,$ $\psi _{t}^{-}=?$. \ The average interbank market
%is $\bar{\imath}_{t}^{f}=?$.
%\end{proposition}
%\end{frame}

\begin{frame}
\frametitle{Liquidity Cost Function $\chi_{t}$ \widehrulefill}
\begin{itemize}
\item Wedge in return to excess reserves and reserves deficits:
\item Marginal Benefit of excess reserves:
\begin{equation*}
  \chi^{+}_{t}=\gamma ^{+}\bar{\imath}_{t}^{f} +(1-\gamma^{+})i^{ior}
%+\left( 1-\gamma
%^{+}\right) r_{t}^{ER}
\end{equation*}%
\item  Marginal cost of reserve deficits:
\begin{equation*}
\chi^{-}_{t}=\gamma ^{-}\bar{\imath}_{t}^{f}+\left( 1-\gamma^{-}\right)i_{t}^{dw}.
\end{equation*}
\item Liquidity Cost Function:
\begin{equation*}
\chi (s)=\left\{
\begin{array}{lll}
\chi_{t}^{+}s \text{ }\text{if \ }s>0\quad  &  & \\
\chi_{t}^{-}s \text{ }\hspace{0.2cm}\text{if \ }s\leq 0\quad  &  &
\end{array}%
\right.
\end{equation*}


%\begin{equation*}
%\chi_{t}(s)=\chi_{t}^{+}s^{+}+\chi_{t}^{-}s^{-}
%\end{equation*}
\end{itemize}
\end{frame}

\linespread{1.5}%
%\begin{frame}
%\frametitle{Banks' Balance Sheet \widehrulefill}
%
%\begin{figure}[tbp]
%\includegraphics[width=300pt, trim=0 100 0 0]{LMMP_BS}
%%\caption{Bank Balance Sheet}
%\end{figure}
%\end{frame}
%
%\begin{frame}
%\frametitle{Banks' Balance Sheet \widehrulefill}
%
%\begin{figure}[tbp]
%\includegraphics[width=300pt, trim=0 100 0 0, page=3]{LMMP_BS}
%%\caption{Bank Balance Sheet}
%\end{figure}
%\end{frame}
%
%\begin{frame}
%\frametitle{Expansion of Balance Sheet \widehrulefill}
%
%\begin{figure}[tbp]
%\includegraphics[width=300pt, trim=0 100 0 0, page=5]{LMMP_BS}
%%\caption{Bank Balance Sheet}
%\end{figure}
%
%\vspace{-2cm}
%
%\hyperlink{Rest}{\beamergotobutton{Rest of the Economy}}
%
%\hypertarget{RestBack}{}
%\end{frame}
%

%%%%
%
%\begin{frame}
%\frametitle{Liquidity Cost function $\chi_{t}$ \widehrulefill}
%
%\begin{itemize}
%\item Wedge in return to excess reserves and reserves deficits:
%\msk
%
%\item Marginal Benefit of excess reserves:
%\begin{equation*}
%\alert<1> \chi^{H}_{t}=\gamma ^{+}r^{FedFunds}+\left( 1-\gamma
%^{+}\right) r_{t}^{ER}
%\end{equation*}%
%\msk
%
%\item  Marginal cost of reserve deficits:
%
%\begin{equation*}
%\alert<1> \chi^{+}_{t}=\gamma ^{-}r^{FedFunds}+\left( 1-\gamma
%^{-}\right)r_{t}^{dw}.
%\end{equation*}
%\msk
%
%\item Liquidity Cost Function:
%\begin{equation*}
%\alert{\chi_{t}(x)=\chi_{t}^{+}x^{-}+\chi_{t}^{-}x^{+}}
%\end{equation*}
%\end{itemize}
%\end{frame}

%



%%%%%%%%%%%%%%%%%%%%%%%%%%%%%%%%%%%%%%%%%%%%%%%%%%%%

%\begin{frame}
%\frametitle{Reserves $M_{t}$ \widehrulefill}
%
%\begin{itemize}
%\item Fixed Aggregate Supply determined by FED: $M0_{t}$
%
%\bigskip \pause
%
%%\item Transferred across banks
%%
%%\begin{itemize}
%%\item Loan withdrawal
%%
%%\item Interbank purchases $\varphi _{t}$
%%\end{itemize}
%
%\bigskip \pause
%
%\item Held as precautionary savings
%
%\begin{itemize}
%\item Avoid liquidity cost: $\chi_{t}(x)$
%\end{itemize}
%\end{itemize}
%\end{frame}

%%%%%%%%%%%%%%%%%%%%%%%%%%%%%%%%%%%%%%%%%%%%%%%%%%%%%%%%
\begin{frame}
\frametitle{Balance Sheet \widehrulefill}
\begin{figure}[tbp]
\includegraphics[width=300pt, trim=0 100 0 0]{GraphsLMMPjb1}
%caption
\end{figure}
\end{frame}

\begin{frame}
\frametitle{Expansion of Lending \widehrulefill}
\begin{figure}[tbp]
\includegraphics[width=300pt, trim=0 100 0 0]{GraphsLMMPjb2}
%caption
\end{figure}
\end{frame}

\begin{frame}
\frametitle{Withdrawal Risk \widehrulefill}
\begin{figure}[tbp]
\includegraphics[width=300pt, trim=0 100 0 0]{GraphsLMMPjb3}
%caption
\end{figure}
\end{frame}

\begin{frame}
\frametitle{Withdrawal Risk \widehrulefill}
\begin{figure}[tbp]
\includegraphics[width=300pt, trim=0 100 0 0]{GraphsLMMPjb4}
%caption
\end{figure}
\end{frame}

\begin{frame}
\frametitle{Withdrawal Risk \widehrulefill}
\begin{figure}[tbp]
\includegraphics[width=300pt, trim=0 100 0 0]{GraphsLMMPjb5}
%caption
\end{figure}
\end{frame}
\begin{frame}
\frametitle{Withdrawal Risk \widehrulefill}
\begin{figure}[tbp]
\includegraphics[width=300pt, trim=0 100 0 0]{GraphsLMMPjb6}
%caption
\end{figure}
\end{frame}
\begin{frame}
\frametitle{Withdrawal Risk \widehrulefill}
\begin{figure}[tbp]
\includegraphics[width=300pt, trim=0 100 0 0]{GraphsLMMPjb7}
%caption
\end{figure}
\end{frame}
\begin{frame}
\frametitle{Withdrawal Risk \widehrulefill}
\begin{figure}[tbp]
\includegraphics[width=300pt, trim=0 100 0 0]{GraphsLMMPjb8}
%caption
\end{figure}
\end{frame}
\begin{frame}
\frametitle{Withdrawal Risk \widehrulefill}
\begin{figure}[tbp]
\includegraphics[width=300pt, trim=0 100 0 0]{GraphsLMMPjb9}
%caption
\end{figure}
\end{frame}
\begin{frame}
\frametitle{Withdrawal Risk \widehrulefill}
\begin{figure}[tbp]
\includegraphics[width=300pt, trim=0 100 0 0]{GraphsLMMPjb10}
%caption
\end{figure}
\end{frame}
\begin{frame}
\frametitle{Withdrawal Risk \widehrulefill}
\begin{figure}[tbp]
\includegraphics[width=300pt, trim=0 100 0 0]{GraphsLMMPjb11}
%caption
\end{figure}
\end{frame}

%\end{document}
%\begin{frame}
%\frametitle{Liquidity Management \widehrulefill}
%
%\begin{figure}[tbp]
%\includegraphics[width=300pt, trim=0 100 0 0]{F_LMMP_BS_WITHDRAWAL}
%\caption{Bank Balance Sheet - Liquid Assets}
%\end{figure}
%\end{frame}
%
%\begin{frame}
%\frametitle{Liquidity Management \widehrulefill}
%
%\begin{figure}[tbp]
%\includegraphics[width=300pt, trim=0 100 0 0, page=2]{F_LMMP_BS_WITHDRAWAL}
%\caption{Bank Balance Sheet - Liquid Assets}
%\end{figure}
%\end{frame}
%
%
%
%\begin{frame}
%\frametitle{Liquidity Management \widehrulefill}
%
%\begin{figure}[tbp]
%\includegraphics[width=300pt, trim=0 100 0 0, page=3]{F_LMMP_BS_WITHDRAWAL}
%\caption{Bank Balance Sheet - Liquid Assets}
%\end{figure}
%\end{frame}
%
%\begin{frame}
%\frametitle{Liquidity Management \widehrulefill}
%
%\begin{figure}[tbp]
%\includegraphics[width=300pt, trim=0 100 0 0, page=4]{F_LMMP_BS_WITHDRAWAL}
%\caption{Bank Balance Sheet - Liquid Assets}
%\end{figure}
%\end{frame}
%


%\begin{frame}
%\frametitle{Liquidity Management \widehrulefill}
%
%\begin{figure}[tbp]
%\includegraphics[width=300pt, trim=0 100 0 0, page=5]{F_LMMP_BS_WITHDRAWAL}
%\caption{Bank Balance Sheet - Liquid Assets}
%\end{figure}
%\end{frame}
%
%\begin{frame}
%\frametitle{Liquidity Management \widehrulefill}
%
%\begin{figure}[tbp]
%\includegraphics[width=300pt, trim=0 100 0 0, page=6]{F_LMMP_BS_WITHDRAWAL}
%\caption{Bank Balance Sheet - Liquid Assets}
%\end{figure}
%\end{frame}
%
%\begin{frame}
%\frametitle{Liquidity Management \widehrulefill}
%
%\begin{figure}[tbp]
%\includegraphics[width=300pt, trim=0 100 0 0, page=7]{F_LMMP_BS_WITHDRAWAL}
%\caption{Bank Balance Sheet - Liquid Assets}
%\end{figure}
%\end{frame}
%
%\begin{frame}
%\frametitle{Liquidity Management \widehrulefill}
%\begin{figure}[tbp]
%\includegraphics[width=300pt, trim=0 100 0 0, page=8]{F_LMMP_BS_WITHDRAWAL}
%\caption{Bank Balance Sheet - Liquid Assets}
%\end{figure}
%\end{frame}
%
%
%\begin{frame}
%\frametitle{Liquidity Management \widehrulefill}
%\begin{figure}[tbp]
%\includegraphics[width=300pt, trim=0 100 0 0, page=9]{F_LMMP_BS_WITHDRAWAL}
%\caption{Bank Balance Sheet - Liquid Assets}
%\end{figure}
%\end{frame}

%
%\begin{frame}
%\begin{figure}[ht]
%\frametitle{Liquidity Cost \widehrulefill} %
%\includegraphics[scale=0.59]{ distribution_penalty_slide0}
%\end{figure}
%\end{frame}
%
%\begin{frame}
%\begin{figure}[ht]
%\frametitle{Liquidity Management \widehrulefill} %
%\includegraphics[scale=0.59]{ distribution_penalty_slide1}
%\end{figure}
%\end{frame}
%
%\begin{frame}
%\begin{figure}[ht]
%\frametitle{$\uparrow$ Deposits \widehrulefill} \centering
%\includegraphics[scale=0.59]{ distribution_penalty_slide2}
%\end{figure}
%\end{frame}
%
%\begin{frame}
%\begin{figure}[ht]
%\frametitle{$\uparrow$ Reserves \widehrulefill} \centering
%\includegraphics[scale=0.59]{ distribution_penalty_slide3}
%\end{figure}
%\end{frame}
%
%\begin{frame}
%\begin{figure}[ht]
%\frametitle{Liquidity Management \widehrulefill} %
%\includegraphics[scale=0.59]{ distribution_penalty_slide1}
%\end{figure}
%\end{frame}
%
%\begin{frame}
%\begin{figure}[ht]
%\frametitle{$\uparrow$ Withdrawal Risk \widehrulefill} %
%\includegraphics[scale=0.59]{ distribution_penalty_slide4}
%\end{figure}
%\end{frame}


%%%%%%%%%%%%%%%%%%%%%%%%%%%%%%%%%%%%%%%%%%%%%%%%%%%%%%%%

\subsubsection{Bank Equity}

%
%\begin{frame}
%\frametitle{Equity $N_{t}$ \widehrulefill}
%
%\begin{itemize}
%\item Dividends constrained by equity
%
%\begin{itemize}
%\item No need to impose non-Ponzi scheme $N_{t}\rightarrow 0.$
%\end{itemize}
%\end{itemize}
%\end{frame}

%%%%%%%%%%%%%%%%%%%%%%%%%%%%%%%%%%%%%%%%%%%%%%%%%

%\begin{frame}
%\frametitle{Timing of Bank Actions \widehrulefill}
%
%\begin{enumerate}
%\item Bank enters period with state ($M_{t}$,$b_{t}$,$d_{t}$)
%
%- Aggregate state $X_{t}$ and prices ($q_{t}$,$r_{t}$) known
%
%\bigskip
%
%\item \textbf{Lending stage:}
%
%- Bank chooses $\left( I_{t},C_{t},\varphi _{t}\right) $
%
%- Coupon payments accrue
%
%\bigskip
%
%\item \textbf{Balancing stage: }$\omega -$ withdrawals
%
%- Banks pay penalty if illiquid - finish with ($M_{t+1}$,$b_{t+1}$,$d_{t+1}$)
%\end{enumerate}
%\end{frame}
%
%%%%%%%%%%%%%%%%%%%%%%%%%%%%%%%%%%%%%%%%%%%%%%%%%

%\section{The FED}
%
%%%%%%%%%%%%%%%%%%%%%%%%%%%%%%%%%%%%%%%%%%%%%%%%%%
%
%\begin{frame}
%\frametitle{FED Balance Sheet \widehrulefill}
%
%\begin{figure}[tbp]
%\includegraphics[width=300pt, trim=0 100 0 0]{F_LMMP_FEDBALANCE}
%\caption{The FED's Balance Sheet}
%\end{figure}
%
%\end{frame}

%%%%%%%%%%%%%%%%%%%%%%%%%%%%%%%%%%%%%%%%%%%%%%%%%

%\begin{frame}
%\frametitle{List of FED Tools \widehrulefill}
%
%\begin{itemize}
%\item Fed-Funds rate and Reserve Requirements:
%
%\begin{itemize}
%\item $\chi ,\rho $ alters FED fund market directly
%\end{itemize}
%
%\medskip
%
%\item OMO: purchase of Loans or Deposits for Reserves \medskip
%
%\item $\kappa $ capital requirements \medskip
%
%\item Laws of Motion for FED's Balance Sheet:%
%\begin{eqnarray*}
%M0_{t+1} &=&M0_{t}+\varphi _{t} \\
%D_{t+1}^{FED} &=&D_{t}^{FED}+\left( 1+r_{t}\right) \varphi _{t}-q_{t}I_{t}+\chi_{t}-trans_{t} \\
%B_{t+1}^{FED} &=&\delta B_{t}^{FED}+I_{t}.
%\end{eqnarray*}
%\end{itemize}
%\end{frame}
%
%
%%%%%%%%%%%%%%%%%%%%%%%%%%%%%%%%%%%%%%%%%%%%%%%%%
%
%\begin{frame}
%\frametitle{Market Clearing Conditions \widehrulefill}
%
%\begin{itemize}
%\item Loan Market Clearing:%
%\begin{equation*}
%q_{t}=\Theta _{t}\left( B_{t+1}-\delta B_{t}+B_{t+1}^{FED}-\delta
%^{T}B_{t}^{FED}\right) ^{\varepsilon }
%\end{equation*}
%
%\item Money Market:%
%\begin{equation*}
%M0_{t+1}-M0_{t}=M_{t+1}-M_{t},
%\end{equation*}
%\end{itemize}
%\end{frame}

%%%%%%%%%%%%%%%%%%%%%%%%%%%%%%%%%%%%%%%%%%%%%%%%%

\begin{frame}[shrink=20]
\frametitle{Value Function - Lending Stage \widehrulefill}
\label{Prob_Lending}


\label{Prob_Lending}
\begin{align*}
V_{t}^{l}\left( b,m,d,f,w\right) & =\max_{\left\{ c,\tilde{b},\tilde{d},%
\tilde{m}\right\} \geq 0}u\left( c\right) +\mathbb{E}\left[ V_{t}^{b}(\tilde{%
b},\tilde{m},\tilde{d},\omega )\right]  & & & & & &  \notag \\
\\
& P_{t}c+\ \tilde{b}+\tilde{m}-\tilde{d} & & & & & &  \notag \\
& =b(1+i_{t}^{b})-d(1+i_{t}^{d})+m(1+i_{t}^{IOR})-(1+\bar{\imath}_{t}^{f})f\
-\left( 1+i_{t}^{dw}\right) w\ -P_{t}T_{t} & & & & & &  \tag{Budget
Constraint} \\
\tilde{d}& \leq \kappa \left( \tilde{b}+\tilde{m}-\ \tilde{d}\right) . & & &
& & &  \tag{Capital Requirement}
\end{align*}


%\begin{eqnarray*}
%V_t^l(b,m,d,f,w ) &=&\max_{\tilde{c},\tilde{m},\tilde{d}}u\left(
%c\right) +\mathbb{E}_{\omega}\left[ V_{t+1}^b(\tilde{c},\tilde{b},\tilde{d} )%
%\right] \\
% \frac{\tilde{d}}{1+i^{D}_t}+M+B &=&D+\tilde{b}+P_t C+\frac{\tilde{m}}{%
%1+i^{ior}_t}+P_tT+F-W  \\
%\tilde{d} &\leq &\kappa \left( \tilde{b}+\tilde{m}-\frac{%
%\tilde{d}}{1+i^{D}_t }\right) ;\tilde{b},\tilde{m},\tilde{d}\geq 0.
%\end{eqnarray*}
\end{frame}



%%%%%%%%%%%%%%%%%%%%%%%%%%%%%%%%%%%%%%%%%%%%%%%%%%%%%%%%%%%%%%%%%%%%%%%
%
\begin{frame}
\frametitle{Value Function - Balancing Stage \widehrulefill}
\begin{align}
V_{t}^{b}(\tilde{b},\tilde{m},\tilde{d},\omega )& =\beta V_{t}^{l}(b^{\prime
},\tilde{m}^{\prime },d^{\prime },f,w) & & & & & &  \notag \\
\text{ }b^{\prime }& =\tilde{b} & & & & & &  \tag{Evolution of Loans} \\%
[0.14in]
d^{\prime }& =\tilde{d}+\omega \tilde{d} & & & & & &
\tag{Evolution
of
Deposits} \\
m^{\prime }& =\tilde{m}-\omega \tilde{d}\left( \frac{1+i_{t+1}^{d}}{%
1+i_{t+1}^{ior}}\right) +f+w & & & & & &  \tag{Evolution of Reserves} \\
s& =\ \tilde{m}+\frac{\omega _{t}\tilde{d}_{t+1}\left( 1+i_{t+1}^{d}\right)
}{1+i_{t+1}^{IOR}}-\rho \tilde{d}_{t+1}\left( 1+\omega \right) \  & & & & & &
\tag{Reserve Balance} \\
m^{\prime }& \geq \rho d^{\prime } & & & & & &  \tag{Reserve Requirement} \\%
[0.14in]
f& =\psi _{t}^{-}s\text{ and }w_{t+1}=\left( 1-\psi _{t}^{-}\right) s\text{
for }s<0 & & & & & &  \\ %\tag{Interbank-Market Transactions} \\[0.14in]
f& =\psi _{t}^{+}s\text{ and }w_{t+1}=0\text{ for }s>0. & & & & & &  \notag
\end{align}
%and in the balancing stage banks solve the following:%
%\begin{eqnarray*}
%V^{b}(\tilde{b},\tilde{m},\tilde{d},\omega;\tilde{X})
%&=&\max_{F,W,m^{\prime }}\beta \left[ V^l(b^{\prime },%
%\tilde{m}^{\prime },d^{\prime },F,W ^{\prime })|\tilde{X}\right] \\
%b^{\prime } &=&\tilde{b} \\
%d^{\prime } &=&\tilde{d}^{\prime }-\omega \tilde{d}^{\prime } \\
%m^{\prime } &=&\tilde{m}-\omega \tilde{d}+ f  +w \\
%m^{\prime } &\geq &\rho d^{\prime }
%\end{eqnarray*}
\end{frame}

%%%%%%%%%%%%%%%%%%%%%%%%%%%%%%%%%%%%%%%%%%%%%%%%%%%%

\begin{frame}
\frametitle{One Value Function \widehrulefill}
Define
\begin{equation*}
e_{t}\equiv \frac{b_{t}(1+i_{t}^{b})+m_{t}(1+i_{t}^{ior})-d_{t}\
(1+i_{t}^{d})-\left( 1+i_{t}^{f}\right) f_{t}-\left( 1+i_{t}^{f}\right)
w_{t}-T_{t}}{P_{t}}.
\end{equation*}%
%\begin{proposition}[Single-State Representation]
%\label{Pp_SingleState}
\begin{align}
V_{t}(e)& =\max_{\left\{ c,\tilde{m},\tilde{b},\tilde{d}\right\} \geq
0}u(c)+\beta \mathbb{E}_{t}\left[ V_{t+1}(e^{\prime })\right] ,
\label{belmman-single}  \notag \\
e& =\ \frac{\ \tilde{b}+\tilde{m}-\tilde{d}}{P_{t}}+c,
\tag{Budget
Constraint} \\
e^{\prime }& =\left( (1+i_{t+1}^{ior})\tilde{m}+\ \left(
1+i_{t+1}^{b}\right) \tilde{b}-\left( 1+i_{t+1}^{d}\right) \tilde{d}+\chi
_{t+1}\left( s\right) \right) \frac{(1-\tau _{t+1})}{P_{t+1}},
\tag{Evolution of Equity} \\
s& =\ \tilde{m}+\frac{\omega _{t}\tilde{d}^{\prime } \left( 1+i_{t+1}^{d}\right)
}{1+i_{t+1}^{IOR}}-\rho \tilde{d}^{\prime }\left( 1+\omega \right) \  & & & & & &
\tag{Reserve Balance} \\
\tilde{d}& \leq \kappa \left( \tilde{b}+\tilde{m}-\tilde{d}\right) .
\tag{Capital Requirement}
\end{align}
%\end{proposition}
\end{frame}



\section{ Bank Value Functions}

%\begin{frame}
%\frametitle{The Aggregate State \widehrulefill}
%
%\begin{itemize}
%\item Governments Policy Path $\left\{ \rho
%_{t},M0_{t},D_{t}^{FED},B_{t}^{FED},\kappa _{t},r^{l}_{t},r^{b}_{t},\bar{L}\right\}
%_{t\geq 0}$ \medskip
%
%\item $\Theta _{t}$ is the slope of demand curve. \medskip
%
%\item $F_{t}$ process for withdrawal risk \medskip
%
%\item Endogenous state variable: \alert{$E_{t}$}
%
%\medskip
%
%\item Aggregate State: \alert{$X_{t}$}
%
%\begin{itemize}
%\item Model recursive in $X_{t}$
%\end{itemize}
%\end{itemize}
%\end{frame}
%
%\begin{frame}
%\frametitle{Central-Bank Balance Sheet \widehrulefill}
%\begin{itemize}
%\setlength{\itemsep}{8pt}
%\item Fed Budget Constraint:
%\begin{eqnarray*}
%\underbrace{p_{t}\left( M_{t+1}^{0}-M_{t}^{0}\right)}_{\text{Change in Money Supply}}= \\
%\underbrace{D_{t+1}^{FED}-D_{t}^{FED}}_{\text{Liquidity Facility}}+\underbrace{q_{t}\left( B_{t+1}^{FED} \right)}_{\text{OMO}}... \\ + \underbrace{T_{t}}_{\text{Transfers}} + \underbrace{B +(r_{t}^{dw}(1-\gamma^{-}) - r_{t}^{ER}(1-\gamma^{+}))M^{0}_{t}}_{\text{Operating Revenues}}
%\label{eq:balance-sheet-fed}
%\end{eqnarray*}
%\pause
%
%\item Fed chooses feasible policy: $\left\{M^{0}_{t+1},B_{t+1}^{FED},D^{FED}_{t+1},i^{dw}_t,r^{ER}_t,T_t\right\}_{t \geq 0}$
%
%\end{itemize}
%\end{frame}

\begin{frame}
\frametitle{Closing the Model \widehrulefill}
\begin{itemize}
\item Loan Market Clears
\begin{equation*}
\frac{B_{t+1}^{d}}{P_{t}}=\Theta _{t}^{b}\left( \frac{1}{(1+i_{t+1}^{b})}%
\frac{P_{t+1}}{P_{t}}\right) ^{\epsilon },\epsilon >0,\Theta _{t}>0,
\label{eq:loan-demand}
\end{equation*}%
\item Deposit Market Clears (today perfectly elastic supply)
\begin{equation*}
\frac{D_{t+1}^{S}}{P_{t}}=\Theta _{t}^{d}\left( \frac{1}{(1+i_{t+1}^{d})}%
\frac{P_{t}}{P_{t+1}}\right) ^{\zeta },\varsigma >0,\Theta _{t}^{d}>0,
\label{eq:Deposit}
\end{equation*}
\end{itemize}
\end{frame}


\begin{frame}
\frametitle{Central Bank Policies: The Fed \widehrulefill}
\begin{itemize}
\item Sets quantity of reserves $M^{Fed}_t$ %and
\item Buys loans $B_{t+1}^{FED}$, place deposits   $D_{t+1}^{FED}$
\item Discount window and interest on reserve rates  $i^{dw},i^{ior}$   % to have agents borrow at this rate need interest chib> bigger.. This would be a shut down interbank market..shock could be idiosyncratic
%\pause
% can add for example maximum amount of discount window loans...needs another market clearing
\item Fed budget constraint:
\begin{eqnarray}
&&M_{t}^{Fed}(1+i_{t}^{ior})+D_{t+1}^{Fed}+\ B_{t+1}^{Fed}+W_{t+1}^{Fed}...
\notag \\
=
&&M_{t+1}^{Fed}+D_{t}^{Fed}(1+i_{t}^{d})+B_{t}^{Fed}(1+i_{t}^{b})+W_{t}^{Fed}(1+i_{t}^{dw})+P_{t}T_{t}.
\notag
\end{eqnarray}%
%\begin{equation*}
%M^{s}_{t}- D^{FED}_t + q_t B^{FED}  =   \frac{M^s_{t+1}}{1+i_{t}^{ior} } +  \frac{D^{FED}_{t+1}}{1+i^D} +T_t  +  B_t^{FED} + \underbrace{r_{t}^{dw}\left( 1-\gamma ^{-}\right) S^{-}}
%_{\substack{ \text{Earnings from }  \\ \text{Discount Loans}}}
%\end{equation*}
\item Stationary equilibrium: constant nominal growth of balance sheet
\item Impulse responses: focus on fixed nominal balance sheet or keep constant inflation \alert<2>{(today)}
%\item At steady state with zero growth in $M^s$ and $i^{ior}=0$,   $P_{t}T_{t}=\chi^{Fed}.$
%\item Constraints on balance sheet (not today)
%\underbrace{p_{t}\left( M_{t+1}^{0}-M_{t}^{0}\right)}_{\text{Change in Money Supply}}= \\
%\underbrace{D_{t+1}^{FED}-D_{t}^{FED}}_{\text{Liquidity Facility}}+\underbrace{q_{t}\left( B_{t+1}^{FED}-\delta
%B_{t}^{FED}\right)}_{\text{OMO}}... \\ + \underbrace{T_{t}}_{\text{Transfers}} + \underbrace{B(1-\delta)+(r_{t}^{dw}(1-\gamma^{-}) - r_{t}^{ior} M^{0}_{t}}_{\text{Operating Revenues}}
%\label{eq:balance-sheet-fed}
%\end{eqnarray*}


%\begin{eqnarray*}
%\underbrace{p_{t}\left( M_{t+1}^{0}-M_{t}^{0}\right)}_{\text{Change in Money Supply}}= \\
%\underbrace{D_{t+1}^{FED}-D_{t}^{FED}}_{\text{Liquidity Facility}}+\underbrace{q_{t}\left( B_{t+1}^{FED}-\delta
%B_{t}^{FED}\right)}_{\text{OMO}}... \\ + \underbrace{T_{t}}_{\text{Transfers}} + \underbrace{B(1-\delta)+(r_{t}^{dw}(1-\gamma^{-}) - r_{t}^{ior} M^{0}_{t}}_{\text{Operating Revenues}}
%\label{eq:balance-sheet-fed}
%\end{eqnarray*}
\end{itemize}
\end{frame}

%%%%%%%%%%%%%%%%%%%%%%%%%%%%%%%%%%%%%%%%%%%%%%%%%%%%%%%%%%%%%%%%%%%%%%%

\begin{frame}
\frametitle{Market Clearing\widehrulefill}
%\begin{itemize}
%\item[(iv)] Markets clear
\begin{align}
\int_{j}b_{t+1}^{j}+B_{t+1}^{Fed}& =B_{t+1}^{d}  \tag{Loans markets clearing}
\\
\int_{j}d_{t}^{j}-D_{t+1}^{Fed}& =D_{t+1}^{S}  \tag{Deposits market clearing}
\\
\int_{j}m_{t+1}^{j}& =M_{t+1}^{Fed}  \tag{Reserves market clearing} \\
\int_{j}f_{t}^{j}& =0  \tag{Interbank markets clearing} \\
\int_{j}w_{t}^{j}& =W_{t+1}^{Fed}  \tag{Discount window market clearing}
\end{align}
%% Fed funds rate is either equal to ior or DW
%\end{itemize}
\end{frame}

%%%%%%%%%%%%%%%%%%%%%%%%%%%%%%%%%%%%%%%%%%%%%%
%%
%%\begin{frame}
%%\frametitle{Characterization \widehrulefill}
%%
%%\begin{enumerate}
%%\item Single endogenous state
%%
%%\begin{itemize}
%%\item $E \equiv qB+pc-D$
%%\end{itemize}
%%
%%\bigskip \pause
%%
%%\item Portfolio Separation Theorem
%%
%%\begin{itemize}
%%\item Dividend-Savings independent of Portfolio Problem
%%
%%\item Portfolio \alert{Non-Linear Liquidity-Adjusted Returns}
%%\end{itemize}
%%
%%\bigskip \pause
%%
%%\item Portfolio Problem %\begin{itemize}
%%%\item Lessons
%%%\end{itemize}
%%\end{enumerate}
%%\end{frame}
%
%%\begin{frame}
%%\frametitle{Solution \widehrulefill}
%%
%%\begin{itemize}
%%\item Law of motion for deposits
%%\begin{equation*}
%%\tilde{d}=D+q \underbrace{\alert{I}}_{\tilde{b}-\delta B}+C+p\underbrace{\alert{\varphi}}_{\tilde{m}-C}-B(1-\delta ).
%%\end{equation*}
%%\pause
%%
%%\item and substitute for $I$ and $\varphi$...%
%%\begin{equation*}
%%\tilde{d}=D+q(\tilde{b}-\delta B)+C+p(\tilde{m}-C)-B(1-\delta )
%%\end{equation*}
%%\pause
%%
%%\item and rearrange terms to obtain...\qquad \qquad
%%\begin{equation*}
%%C+p\tilde{m}+\tilde{b}-\tilde{d}=\underbrace{pC+\left( q\delta +(1-\delta
%%)\right) B-D}_{\alert<4>{E}}.
%%\end{equation*}
%%\pause
%%
%%\item We can collapse all state-variables into one: \alert<4>{E!}
%%\end{itemize}
%%\end{frame}
%
%\begin{frame}
%\frametitle{Characterization \widehrulefill}
%
%\begin{proposition}[Single-State ]
%We have
%\begin{equation*}
%V^{l}(C,B,D )=V^{l}(E )
%\end{equation*}
%\begin{equation*}
%E\equiv pC+q\delta B+B(1-\delta )-D.
%\end{equation*}
%\end{proposition}
%
%\pause
%\bigskip
%
%
%\end{frame}

%%%%%%%%%%%%%%%%%%%%%%%%%%%%%%%%%%%%%%%%%%%%%%%%%%%%%%%%%%%%%%%%%%%%%%%

%\begin{frame}
%\frametitle{Bank Portfolio Problem \widehrulefill}
%
%\begin{itemize}
%\item Four Returns:
%\msk
%
%\pause
%
%\begin{itemize}
%\item Return on Loans:
%\begin{equation*}
%R_{t}^{B} \equiv \frac{\delta q_{t+1}+(1-\delta )}{q_{t}},
%\end{equation*}
%\pause
%
%\item Return on Reserves:%
%\begin{equation*}
%R_{t}^{C}\equiv \left( \frac{p_{t+1}}{p_{t}}\right)
%\end{equation*}
%\pause
%
%\item Return on Deposits:
%\begin{equation*}
%R^{D}_{t} \equiv 1
%\end{equation*}
%\pause
%
%\item Liquidity Cost:
%\begin{equation*}
%R^{\chi}\left(\alert{w_{d},w_{c}},\omega^{\prime}\right)  \equiv \chi \left( \left(\rho +(1-\rho) \alert{\omega^{\prime}} \right) w_{d}-w_{c}\right)
%\end{equation*}
%\end{itemize}
%\end{itemize}
%\end{frame}

%%%%%%%%%%%%%%%%%%%%%%%%%%%%%%%%%%%%%%%%%%%%%%%
%
%\begin{frame}
%
%\frametitle{Bank Portfolio Problem \widehrulefill}
%
%\begin{itemize}
%\item Effects of MP captured by $\Omega \left( X \right)$
%
%\item $\Omega \left(X\right)$ certainty equivalent portfolio:
%
%\begin{equation*}
%\max_{\{\bar{b},w_{d},w_{c}\} \in \mathbb{R}_{+}^{3}} \left( \mathbb{E}_{\omega^{\prime}} [\left( R^{B} \bar{b}+R^{C} w_{c} - R^{D} w_{d} - R^{\chi}(w_{d},w_{c},\omega^{\prime}) \right)^{1-\gamma}] \right)^{\frac{1}{1-\gamma}}
%\end{equation*}
%subject to,
%\begin{eqnarray*}
%1 &=& \bar{b}+w_{c}-w_{d} \\
%w_{d} &\leq &\kappa \left(\bar{b}+w_{c}-w_{d}\right) \\
%\end{eqnarray*}
%
%\item Original Policies: $[\tilde{d},\tilde{b},\tilde{m}]=[w_{d},\bar{b},w_{c}]\cdot E \cdot (1-c)$
%
%\end{itemize}
%
%\end{frame}

%%%%%%%%%%%%%%%%%%%%%%%%%%%%%%%%%%%%%%%%%%%%%%%%%
%
%\begin{frame}
%\frametitle{Bank Portfolio Problem \widehrulefill}
%
%\begin{itemize}
%\item { $\Omega _t $ also }
%
%{\footnotesize \hspace{-1cm}%
%\begin{equation*}
%\max_{\left\{ w_{d},w_{c}  \in \mathbb{R}_{+}^{2} \right\}}\mathbb{CE}_{\mathbb{\omega }^{\prime
%}} [ \underset{\text{Loan Return}}{\underbrace{R^{B}}}\underset{%
%\text{Opportunity Cost}}{\underbrace{- \left( R^{B}-R^{C}\right)w_{c} }}%
%+\underset{\text{Arbitrage}}{\underbrace{\left( R^{B}-R^{D}\right) }}w_{d}-%
%\underset{\text{Liquidity Cost}}{\underbrace{R^{\chi }\left(
%w_{d},w_{c}\right) }} ]
%\end{equation*}
%subject to,%
%\begin{eqnarray*}
%w_{d} &\leq &\kappa \\
%\end{eqnarray*}
%}
%\end{itemize}
%\end{frame}

%%%%%%%%%%%%%%%%%%%%%%%%%%%%%%%%%%%%%%%%%%%%%%%
%
%\begin{frame}
%\frametitle{Liquidity Premium \widehrulefill}
%\begin{itemize}
%\item Banks require a higher return on loans:
%\bigskip
%\begin{equation*}
%\underset{\text{Liquidity Premium}} {\underbrace{R_{t+1}^{B}-R_{t+1}^{C}}} = \underset{\text{First-order effect}} {\underbrace{E_{t}\chi ^{\prime }}}+
%\underset{\text{Liquidity Risk Premium}} {\underbrace{  \frac{Cov(m_{t+1},\chi_{t+1})}{E_{t}m_{t+1}}}}
%\end{equation*}
%\item $ m_{t+1}$ is the SDF \bigskip
%\item Monetary policy can affect lending by altering the liquidity premium
%\end{itemize}
%\end{frame}

%%%%%%%%%%%%%%%%%%%%%%%%%%%%%%%%%%%%%%%%%%%%%%%%
%
%\begin{frame}
%\frametitle{Liquidity Management \widehrulefill}
%
%\begin{figure}[tbp]
%\includegraphics[width=300pt, trim=0 100 0 0]{LMMP_Portfolio2}
%\caption{Portfolio Problem}
%\end{figure}
%\end{frame}
%
%%%%%%%%%%%%%%%%%%%%%%%%%%%%%%%%%%%%%%%%
%
%\begin{frame}
%\frametitle{Liquidity Management and Monetary Policy \widehrulefill}
%
%\begin{itemize}
%\item Monetary Policy Instruments
%
%\begin{itemize}
%\item Discount window and borrowing rates: $(r^{l}_{t},r^{b}_{t})$
%
%\item Reserve requirements $\rho_{t}$
%
%\item Long-Term Loans: $M0_{t}$
%
%\item Open-market operations: $(b_{t},M_{t})$
%\end{itemize}
%\end{itemize}
%\end{frame}
%
%%%%%%%%%%%%%%%%%%%%%%%%%%%%%%%%%%%%%%%%%
%
%\begin{frame}
%\frametitle{Liquidity Management \widehrulefill}
%
%\begin{figure}[tbp]
%\includegraphics[width=300pt, trim=0 100 0 0]{LMMP_Portfolio2}
%\caption{Portfolio Problem}
%\end{figure}
%\end{frame}
%
%%%%%%%%%%%%%%%%%%%%%%%%%%%%%%%%%%%%%%%
%
%\begin{frame}
%\frametitle{Liquidity Management \widehrulefill}
%
%\begin{figure}[tbp]
%\includegraphics[width=300pt, trim=0 100 0 0, page=2]{LMMP_Portfolio2}
%\caption{Portfolio Problem}
%\end{figure}
%\end{frame}
%
%\begin{frame}
%\frametitle{Liquidity Management \widehrulefill}
%
%\begin{figure}[tbp]
%\includegraphics[width=300pt, trim=0 100 0 0, page=3]{LMMP_Portfolio2}
%\caption{Portfolio Problem}
%\end{figure}
%\end{frame}

%%%%%%%%%%%%%%%%%%%%%%%%%%%%%%%%%%%%%%%%%%%%%%%%%%%%%




\begin{frame}
\begin{figure}[ht]
\frametitle{Liquidity Cost \widehrulefill}
\includegraphics[scale=0.59]{distribution_penalty_slide0}
\end{figure}
\end{frame}

\begin{frame}
\begin{figure}[ht]
\frametitle{Liquidity Management \widehrulefill}
\includegraphics[scale=0.59]{distribution_penalty_slide1}
\end{figure}
\end{frame}

\begin{frame}
\begin{figure}[ht]
\frametitle{$\uparrow$ Deposits \widehrulefill}
\centering
\includegraphics[scale=0.59]{distribution_penalty_slide2}
\end{figure}
\end{frame}
\begin{frame}
\begin{figure}[ht]
\frametitle{$\uparrow$ Reserves \widehrulefill}
\centering
\includegraphics[scale=0.59]{distribution_penalty_slide3}
\end{figure}
\end{frame}

\begin{frame}
\begin{figure}[ht]
\frametitle{Liquidity Management \widehrulefill}
\includegraphics[scale=0.59]{distribution_penalty_slide1}
\end{figure}
\end{frame}

\begin{frame}
\begin{figure}[ht]
\frametitle{$\uparrow$ Withdrawal Risk \widehrulefill}
\includegraphics[scale=0.59]{distribution_penalty_slide4}
\end{figure}
\end{frame}


\begin{frame}
\begin{figure}[ht]
\frametitle{Equity Growth \widehrulefill}
\includegraphics[scale=0.55]{eq_growth_slide}
\end{figure}
\end{frame}


 \linespread{1.}
\begin{frame}[shrink=0.04]
%\frametitle{Characterization \widehrulefill}

\begin{proposition}[Homogeneity and Portfolio Separation]
\end{proposition}
\begin{enumerate}[(i)]
\item The value function $V_t(b,m,d)$ satisfies \label{Pp_Homogeneity}%
\begin{equation*}
V_{t}(e)=v_{t}\left( e\right) ^{1-\gamma },  \label{V}
\end{equation*}

\item where $v\left( \cdot \right) $ satisfies
\begin{align*}
& \Omega _{t}\equiv \max_{\left\{ \bar{b},\bar{m},\bar{d}\right\} \geq
0}\left\{ \mathbb{E}_{\omega }\left[ R_{t}^{b}\bar{b}+\ R_{t}^{m}\bar{m}-\ \
R_{t}^{d}\bar{d}+\chi (\bar{m},\bar{d},\omega )\right] ^{1-\gamma }\right\}
^{\frac{1}{^{1-\gamma }}}, \notag \\
& \bar{b}+\bar{m}-\bar{d}=1,  \notag \\
& \bar{d}\leq \kappa \left( \bar{b}+\bar{m}-\bar{d}\right) .  \notag
\end{align*}

%\item policy functions for $\{ \tilde{b},\  \tilde{m},\  \tilde{d}\}$  can be recovered as
 \item
 \begin{eqnarray*}
\tilde{b}_{t+1}^{\prime }(e_{t}) &=&P_{t}\bar{b}_{t}(1-\bar{c}\ )e_{t}, \\
\tilde{m}_{t+1}(e_{t}) &=&P_{t}\bar{m}_{t}(1-\bar{c}\ )e_{t}, \\
\tilde{d}_{t+1}(e_{t}) &=&P_{t}\bar{d}_{t}\ (1-\bar{c})e_{t}.
\end{eqnarray*}
%\begin{equation*}
%\tilde{b}^{\prime }=\frac{\left( 1-P_t )(\tau +\tilde{c}\right) }{q_{b}_t }%
%w_bE,\  \tilde{m}=(1+i^{ior}_t )\left( 1-P_t )(\tau +\tilde{c}\right) w_mE,\
%\tilde{d}=(1+i^{D}_t )\left( 1-P_t )(\tau +\tilde{c}\right) w_dE
%\end{equation*}

\item $\bar{c}_{t},v_t$ are given by
\begin{equation*}
v_{t}=\frac{1}{1-\gamma }\left[ 1+\left( \beta (1-\gamma )\Omega
_{t}^{1-\gamma }v_{t+1}\right) ^{\frac{1}{\gamma }}\right] ^{\gamma }.
\end{equation*}

\begin{equation*}
\bar{c}_{t}\ =\frac{1}{1+\left[ \beta (1-\gamma )v_{t+1}\Omega
_{t}{}^{1-\gamma }\right] ^{1/\gamma }}.  \label{eq:c1}
\end{equation*}
%\item and the value of the bank
%\begin{equation*}
%\upsilon _t =\frac{1}{1-\gamma }\left[ 1+\left( \beta (1-\gamma
%)\Omega ^{\ast }_t ^{1-\gamma }\mathbb{E}\left[ v\left(
%X^{\prime }\right) |X\right] \right) ^{\frac{1}{\gamma }}\right] ^{\gamma }.
%\end{equation*}
\end{enumerate}
%\end{proposition}
\end{frame}


\begin{frame}
\frametitle{Liquidity Premium\widehrulefill}

%\item Banks require a higher return on loans relative to reserves: \bigskip
\begin{eqnarray}
\ \underset{\text{ Liquidity Premium}}{\underbrace{R^{b}-R^{m}}} &=&\underset{\text{First-order liquidity premium}}{\underbrace{\mathbb{E}%
_{\omega }\left[ \frac{\partial \chi \left( \bar{d},\bar{m},\omega \right) }{%
\partial \bar{m}}\right] }}+\underset{\text{Liquidity risk premium}}{%
\underbrace{\mathbb{E}_{\omega }\frac{\mathbb{COV}_{\omega }\left[ \left(
R^{e}\right) ^{-\gamma },\frac{\partial \chi \left( \bar{d},\bar{m},\omega
\right) }{\partial \bar{m}}\right] }{\mathbb{E}_{\omega }\left[ \left(
R^{e}\right) ^{-\gamma }\right] }}}\ .  \notag
\end{eqnarray}

\bigskip
\begin{equation}
\underset{\text{External finance premium}}{\underbrace{R^{b}-R^{d}}=}%
\underset{\text{Deposit Liquidity Cost}}{\underbrace{\frac{\mathbb{E}%
_{\omega }\left[ \left( R_{\omega }^{e}\right) ^{-\gamma }\cdot \frac{%
\partial \chi \left( \bar{d},\bar{m},\omega \right) }{\partial \bar{d}}%
\right] }{\mathbb{E}_{\omega }\left[ \left( R^{e}\right) ^{-\gamma }\right] }%
}}+\underset{\text{Collateral term}}{\underbrace{\frac{\mu }{\mathbb{E}%
_{\omega }\left[ \left( R_{\omega }^{e}\right) ^{-\gamma }\right] }}}
\notag
\end{equation}

%\item $m_{t+1}$ is the SDF \bigskip
\pause
%\begin{proposition}[Full Satiation]
Banks are\ satiated with reserves if
\begin{enumerate}[ (i)]
\item The Fed pays interest on reserves such that $%
i_{t}^{ior}=i_{t}^{b}$
\item The Fed pays interest on reserves such that $%
i_{t}^{ior}=i_{t}^{D}$, and $\kappa =\infty $
\item The Fed sets the discount window rate to $i_{t}^{dw}=0$
\end{enumerate}

\end{frame}

%%%%%%%%%%%%%%%%%%%%%%%%%%%%%%%%%%%%%%%%%%%%%%%%%%%%%%
%
%\begin{frame}
%
%\frametitle{Role of Uncertainty \widehrulefill}
%
%\begin{itemize}
%
%%\item \textbf{Case I:} With Risk-Neutral Banks
%%
%%\begin{itemize}
%%\item $L^{\ast }$ reserves over deposit ratio
%%
%%\item Optimality:
%%
%%\begin{equation*}
%%\underset{\text{Arbitrage on Loans}}{\underbrace{\left( R^{B}-R^{D}\right) }}%
%%=\underset{\text{Optimal Liquidity Ratio Cost}}{\underbrace{%
%%\left(R^{B}-R^{C}\right) L^{\ast }+\alert<1>{\mathbb{E}_{\omega }\left[R_{c}^{\chi} \right] }}}.
%%\end{equation*}
%%
%%\item \textbf{Lesson 2:} Role of kink
%%
%%\end{itemize}
%%
%%\medskip \pause
%
%\item No uncertainty:
%
%\begin{equation*}
%\left( R_{t}^{B}-R_{t}^{D}\right) =\left( R_{t}^{B}-R_{t}^{C}\right) \underbrace{\rho}_{\text{Reserve Req.}} .
%\end{equation*}
%\bigskip
%\item \textbf{Lesson 3:} No effect from corridor rates
%
%\end{itemize}
%
%\end{frame}
%
%%%%%%%%%%%%%%%%%%%%%%%%%%%%%%%%%%%%%%%%%%%%%%%%%%%%%%%%%%%%%%%%%%%%%%%%%%%%%%%%%%
%
%
%\begin{frame}
%\frametitle{Polar Case III \widehrulefill}
%
%\begin{itemize}
%
%\item Zero-Lower Bound: $\chi =0$
%
%\begin{itemize}
%\item $R_{t}^{B}=R_{t}^{C}>R^{D}$ if $\kappa_{t}>0$
%
%\item $R_{t}^{B}=R_{t}^{C}=R^{D}$ if $\kappa_{t}=0$
%
%\end{itemize}
%\bigskip
%\item \textbf{Lesson 4:} Monetary Policy is ``maxed'' out
%
%\bigskip
%\item \textbf{Lesson  5:} Individual Portfolio Indeterminacy
%
%\medskip
%
%\pause
%
%\item \alert{Rates are not zero though!}
%
%\end{itemize}
%
%\end{frame}


\begin{frame}
{\huge {Quantitative Exercise} }
\end{frame}

\begin{frame}
\frametitle{Calibration Strategy \widehrulefill}
\begin{itemize}
 \setlength{\itemsep}{7pt}
\msk
\item $F_{t}$ approximated with a logistic distribution
\begin{itemize}
\item Cross-sectional distribution of deposits growth rates (Call Reports)  \hyperlink{distribution}{\beamergotobutton{distribution}}
\end{itemize}
\msk
%\item Maturity $\delta=0$
\msk
\item Loan-demand elasticity $=1.8$
\begin{itemize}
\item Bassett,  Chosak, Driscoll,  and Zakrajsek (2013)
\end{itemize}
\msk
%\item Period is quarter
\end{itemize}

\end{frame}



\begin{frame}
\frametitle{Calibration \widehrulefill}
\renewcommand{\arraystretch}{1.1}
\begin{table}
%\caption{Calibration}
%\medskip{}
\footnotesize{
\begin{tabular}{lll}
\toprule
%\footnotesize{
 & Value  & Source/Target\tabularnewline
\midrule
%\midrule
Capital requirement  & $\kappa\hspace{0.54cm}=10$  & Regulatory parameter\tabularnewline
Discount factor  & $\beta\hspace{0.52cm}=0.993$  & Dividend ratio = 8 \%\tabularnewline
Risk aversion  & $\gamma\hspace{0.52cm}=1$  & Constant dividend-equity ratio\tabularnewline
Reserve requirement  & $\rho\hspace{0.53cm}=0.1$  & Regulatory parameter\tabularnewline
Deposit supply intercept  & $\Theta^{d}\hspace{0.27cm}=9.6$  & Annual deposit rate = 1\%\tabularnewline
Loan demand intercept  & $\Theta^{b}\hspace{0.275cm}=10.4$  & Unit steady state equity \tabularnewline
Discount window rate (annual)  & $i^{dw}\hspace{0.275cm}=6\%$  & 2006 value\tabularnewline
Interest on reserves (annual)  & $i^{ior}\hspace{0.15cm}=0\%$  & 2006 value\tabularnewline
Bargaining parameter  & $\eta\hspace{0.475cm}=0.5$  & Baseline value\tabularnewline
Inflation  & $g\hspace{0.475cm}=2\%$  & Long-run inflation target\tabularnewline
\midrule
Matching friction  & $\lambda\hspace{0.45cm}=2.1$  & DW loans to reserves $W/M=2%\ensuremath{\%}
$\tabularnewline
Volatility  & $\sigma\hspace{0.45cm}=0.04$  & Reserve-balance distribution\tabularnewline
Loan demand deposit supply elasticities  & $\zeta=-\epsilon=25$  & Bank credit response to policy rate \tabularnewline
\bottomrule
\end{tabular}\label{tab:calibration}}
\end{table}
\end{frame}


%\end{frame}

\begin{frame}
\frametitle{Quantitative Application \widehrulefill}

\begin{itemize}
\item Why are banks not lending when holding so many reserves?
\bigskip
\pause
\item Five Hypotheses:
\begin{enumerate}
\item Equity Losses
\item Tighter Capital Requirements
\item Precautionary Motive (Interbank Market Shutdown)
\item Weaker Loan Demand
\item Interest on Reserves
\end{enumerate}
\bigskip \pause

\item \textit{Approach}

\begin{itemize}

\item First: Transitional Dynamics after individual shocks

\item Second: Fit calibrated sequences of shocks

\end{itemize}
\end{itemize}
\end{frame}

\end{document}
\include{slides_IRF}

%\include{quantitative_slides} %
%
%

\begin{frame}
\label{lastslide} \frametitle{Conclusions \widehrulefill}

\begin{itemize}
  \setlength{\itemsep}{14pt}

\item We developed a new dynamic macroeconomic model of banks liquidity management and monetary policy
% \item Liquidity management problem leads to endogenous liquidity premium

\item Transmission of monetary policy through banking system

\item Uncover shocks affecting banks and the economy during crisis

\begin{itemize}
\item Precautionary motive played important early role

\item ``Demand'' shock is the key force to explain persistence
\end{itemize}
\item Other possible applications: shadow banking, Great Depression, currency pegs, macroprudential
%\item Current project: international liquidity crises %open-economy version
\end{itemize}
\end{frame}

\begin{frame}

EXTRAS

\end{frame}

\begin{frame}[label=hist0]
\begin{figure}[t!]
\begin{center}
%\includegraphics[width=500pt,scale=0.1]{../Empirical Tests/F_HistGrowth} %\newline
\vspace{-2cm}
\caption{Histogram of Deviations from Cross-Sectional Mean Growth Rates for
Total Deposits. For every bank-quarter observation, the histogram reports
frequencies for deviations of the growth rate of total deposits relative to
the cross-sectional average growth of total deposits in a given quarter.}
\label{F_HistGrowth}
\end{center}
  \hyperlink{distribution}{\beamergotobutton{distribution}}
\end{figure}
\end{frame}


\end{document}
