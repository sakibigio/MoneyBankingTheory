%2multibyte Version: 5.50.0.2960 CodePage: 1252

\documentclass{article}
%%%%%%%%%%%%%%%%%%%%%%%%%%%%%%%%%%%%%%%%%%%%%%%%%%%%%%%%%%%%%%%%%%%%%%%%%%%%%%%%%%%%%%%%%%%%%%%%%%%%%%%%%%%%%%%%%%%%%%%%%%%%%%%%%%%%%%%%%%%%%%%%%%%%%%%%%%%%%%%%%%%%%%%%%%%%%%%%%%%%%%%%%%%%%%%%%%%%%%%%%%%%%%%%%%%%%%%%%%%%%%%%%%%%%%%%%%%%%%%%%%%%%%%%%%%%
\usepackage{amsfonts}
\usepackage{setspace}
\usepackage{amsmath,amsthm}
\usepackage{amssymb}
\usepackage{graphicx}
\usepackage[margin=1in]{geometry}

\setcounter{MaxMatrixCols}{10}
%TCIDATA{OutputFilter=LATEX.DLL}
%TCIDATA{Version=5.50.0.2960}
%TCIDATA{Codepage=1252}
%TCIDATA{<META NAME="SaveForMode" CONTENT="1">}
%TCIDATA{BibliographyScheme=Manual}
%TCIDATA{Created=Wednesday, May 21, 2014 12:50:09}
%TCIDATA{LastRevised=Wednesday, April 06, 2016 16:13:25}
%TCIDATA{<META NAME="GraphicsSave" CONTENT="32">}
%TCIDATA{<META NAME="DocumentShell" CONTENT="Standard LaTeX\Blank - Standard LaTeX Article">}
%TCIDATA{CSTFile=40 LaTeX article.cst}

\newtheorem{theorem}{Theorem}
\newtheorem{acknowledgement}[theorem]{Acknowledgement}
\newtheorem{algorithm}[theorem]{Algorithm}
\newtheorem{axiom}[theorem]{Axiom}
\newtheorem{case}[theorem]{Case}
\newtheorem{claim}[theorem]{Claim}
\newtheorem{conclusion}[theorem]{Conclusion}
\newtheorem{condition}[theorem]{Condition}
\newtheorem{conjecture}[theorem]{Conjecture}
\newtheorem{corollary}[theorem]{Corollary}
\newtheorem{criterion}[theorem]{Criterion}
\newtheorem{definition}[theorem]{Definition}
\newtheorem{example}[theorem]{Example}
\newtheorem{exercise}[theorem]{Exercise}
\newtheorem{lemma}[theorem]{Lemma}
\newtheorem{notation}[theorem]{Notation}
\newtheorem{problem}[theorem]{Problem}
\newtheorem{proposition}[theorem]{Proposition}
\newtheorem{remark}[theorem]{Remark}
\newtheorem{solution}[theorem]{Solution}
\newtheorem{summary}[theorem]{Summary}
\newenvironment{proof}[1][Proof]{\noindent\textbf{#1.} }{\ \rule{0.5em}{0.5em}}
\doublespace
\input{tcilatex}
\begin{document}


\begin{center}
\textbf{Chapter 1: Aggregation under Epstein-Zin Preferences \\[0pt]
Models with Financial Frictions}
\end{center}

\section{Overview}

In this notes I derive the optimal portfolio holdings of assets and
consumption policies in a model in which agents feature Epstein-Zin utility.
The virtue of this model is that optimal policies are linear in wealth
despite having a special form of borrowing constraints. We will see that the
model can be accomodated to allow for \textit{natural borrowing constraints
or other type of financial constraints coming from Moral Hazard, Adverse
Selection or dynamic constraints}. This property entails that aggregation is
immediate thanks to the famous Gorman's aggregation theorem. With this,
prices may be determined without carrying non-finite state variables such as
entire distrubitions. Epstein-Zin's preferences nest more standard
preferences such as CRRA ---and therefore log utility--- which makes this a
good starting point. It is also a great device to isolate two forces: risk
aversion and the desire to smooth consumption over time.

This class of models accommodate well for various situations, as I will try
to make clear along these lectures. I describe two models in this section.
In the first part, I describe EZ preferences, and then solve for the model
presented by Angeletos $\left( RED,\text{ }2007\right) $. The tools used in
this section will be used over and over again throughout these lectures.

\section{Notation}

I denote a variable x at a given point in time t and history of events $%
s^{t} $ via $x_{t}\left( s^{t}\right) .$ When I simply write $x_{t}$ it is
clear that the variable is definied for a given history. I append a
supperscript i to refer to the choice of a particular individual.

Also, I denote by $u_{t}^{i}$ the utility of an indivual by time t and $%
v_{t}^{i}$ the consumption equivalent.

\section{Epstein Zin-Preferences}

I follow notation used by Angeletos $\left( RED,\text{ }2007\right) $.
Preferences for agent $i$ are given by:%
\begin{equation*}
u_{t}^{i}=U\left( c_{t}^{i}\right) +\beta \cdot U\left( \mathbb{CE}%
_{t}\left( U^{-1}\left( u_{t+1}^{i}\right) \right) \right)
\end{equation*}%
where
\begin{equation}
\mathbb{CE}_{t}\mathcal{=}\Upsilon ^{-1}\left( \mathbb{E}_{t}\Upsilon \left(
\cdot \right) \right)  \label{1}
\end{equation}%
where the expectation is taken over time $t$ information. The expectation
operator $\mathbb{E}_{t}$ contains all relevant information up to time t.

The term $\mathbb{CE}_{t}$ refers to the certainty equivalent utility with
respect to a CRRA, $\Upsilon ,$ transformation:

\begin{equation*}
\Upsilon \left( c\right) =\frac{c^{1-\gamma }}{1-\gamma }\text{ and }U\left(
c\right) =\frac{c^{1-1/\theta }-1}{1-1/\theta }.
\end{equation*}%
The transformation $\Upsilon \left( c\right) $ characterizes the relative
risk aversion through the parameter $\gamma \geq 0$. If $\gamma =0,$ then
agent is risk-neutral. This specification embeds several polar cases. For
example:

\textbf{Kinghtian Agents.} Suppose risk aversion approaches infinity. Then:

\begin{equation*}
\lim_{\gamma \rightarrow \infty }\mathbb{CE}_{t}\left[ W\left( s\right) %
\right] =\lim_{\gamma \rightarrow \infty }\left( \sum_{s\in \mathcal{S}}\pi
\left( s\right) \left[ W\left( s\right) \right] ^{1-\gamma }\right) ^{\frac{1%
}{1-\gamma }}=\min_{s}\left\{ W\left( s\right) \right\}
\end{equation*}%
so the agent's preferences become \textquotedblleft
Knightian\textquotedblright\ ---he cares only about the worst-case scenario,
regardless of the probability. Recall notes on CES class. These are the
preferences used in Caballero and Farhi (2014).

\textbf{Log-Utility. }When $\theta \rightarrow 1,$ we have that:
\begin{equation*}
\lim_{\theta \rightarrow 1}\frac{c^{1-1/\theta }-1}{1-1/\theta }=\log \left(
c\right) .
\end{equation*}%
Recall that the minus 1 is important. Often, people forget to include that
term.

\textbf{CRRA. }Set $\gamma =\frac{1}{\theta }$ and insert the condition of
utility and one obtains:%
\begin{eqnarray*}
\Upsilon ^{-1}\left( u\right) &=&\left( \left( 1-\gamma \right) u\right) ^{%
\frac{1}{1-\gamma }} \\
U\Upsilon ^{-1}\left( u\right) &=&u\text{ and by analogy,} \\
\Upsilon U^{-1}\left( u\right) &=&u.
\end{eqnarray*}%
Hence, for this particular case we obtain:%
\begin{eqnarray*}
u_{t}^{i} &=&U\left( c_{t}^{i}\right) +\beta \cdot U\Upsilon ^{-1}\left(
E_{t}\Upsilon \left( U^{-1}(u_{t+1}\right) \right) \\
u_{t}^{i} &=&U\left( c_{t}^{i}\right) +\beta E_{t}\left[ u_{t+1}\right] \\
&=&\sum_{t=0}^{\infty }\beta ^{t}U\left( c_{t}^{i}\right) .
\end{eqnarray*}%
and we are back in standard expected present discounted utility. The nice
feature of Epstein-Zin preferences is that it allows us to decompose
risk-aversion with elasticity of substitution, as should be clear very soon.
Moreover, it may be very convienient to study polar cases that yield closed
form solutions.

Through $\theta >0,$ $U\left( c\right) $, characterizes intertemporal
substitution. Let's derive the intertemporal elasticity of substitution to
gain further insights.

\subsection{A two-period Model}

Assume that the agent's utility $u_{t+1}$ is given by the utility evaluated
at his wealth at t+1.$~$That is, assume that the agent consume's his wealth
at t+1. Thus, $u_{t+1}=U\left( c_{t+1}\right) =U\left( W_{t+1}\right) .$

Then, assume that the agent's budget constraint is the following:

\begin{equation*}
c_{t}+a_{t+1}=W_{t}.
\end{equation*}%
Where $a_{t+1}$ are the savings carried out to t+1. Then, let W$%
_{t+1}=R_{t}a_{t}$ and assume for the time being that there's a constant
return $R_{t}.$ We are interested in computing how $c_{t}$ varies with $%
R_{t}.$ Thus, since there is no risk, the problem becomes:

\begin{equation*}
\max_{c_{t}^{i}}U\left( c_{t}^{i}\right) +\beta \cdot U\left( R_{t}\left(
W_{t}-c_{t}^{i}\right) \right)
\end{equation*}%
---we can do this because the certainty equivalent operator becomes
constant. The first-order condition of the problem is the Euler equation:

\begin{equation*}
U^{\prime }\left( c_{t}^{i}\right) =\beta R_{t}U^{\prime }\left( R_{t}\left(
W_{t}-c_{t}^{i}\right) \right) .
\end{equation*}

Suppose we can write $c_{t}^{i}=\varsigma _{t}W_{t}.$ Thus, the equation
above is:

\begin{eqnarray*}
U^{\prime }\left( \varsigma _{t}W_{t}\right) &=&\beta R_{t}U^{\prime }\left(
R_{t}\left( W_{t}-\varsigma _{t}W_{t}\right) \right) \\
&=&\beta R_{t}U^{\prime }\left( R_{t}\left( 1-\varsigma _{t}\right)
W_{t}\right) .
\end{eqnarray*}%
Then, inverting $U^{\prime }$, we obtain:

\begin{equation*}
\varsigma _{t}W_{t}=\Psi \left( \beta ,R_{t}\right) \left( 1-\varsigma
_{t}\right) W_{t}
\end{equation*}%
for the term $\Psi \left( \beta ,R_{t}\right) $ where%
\begin{eqnarray*}
\Psi \left( \beta ,R_{t}\right) &\equiv &\left( \beta R_{t}\right) ^{-\theta
}R_{t} \\
&=&\beta ^{-\theta }R_{t}^{1-\theta }.
\end{eqnarray*}

W can solve for $\varsigma _{t}W_{t}$ to yield:

\begin{equation*}
c_{t}^{i}=\varsigma _{t}W_{t}=\left[ \frac{\Psi \left( \beta ,R_{t}\right) }{%
1+\Psi \left( \beta ,R_{t}\right) }\right] W_{t}.
\end{equation*}%
Thus, we have the following recursion for the marginal propensity to consume:

\begin{equation*}
\varsigma _{t}=\frac{R_{t}^{1-\theta }}{\beta ^{\theta }+R_{t}^{1-\theta }}.
\end{equation*}%
We arrange this expression to obtain:

\begin{equation*}
\frac{1}{\varsigma _{t}}=\frac{\beta ^{\theta }+R_{t}^{1-\theta }}{%
R_{t}^{1-\theta }}.
\end{equation*}%
\textbf{Observations.} Some observations emerge.

\textit{Log Case. }When $\theta =1,$ we have that $\beta ^{\theta
}R_{t}^{1-\gamma }=\beta .$ The result is exactly the same as above, but $%
\varsigma _{t}=\left( 1-\beta \right) $ is true everywhere, even away from
the steady state.

\textbf{Intertemporal-Elasticity of Substitution I. }From
\begin{equation*}
\varsigma _{t}=\frac{R_{t}^{1-\theta }}{\beta ^{\theta }+R_{t}^{1-\theta }}
\end{equation*}%
we will try to compute the IES with respect to $R_{t}.$ Taking derivatives
w.r.t. $R_{t}$ :

\begin{equation*}
\frac{\partial \varsigma _{t}}{\partial R_{t}}=\frac{\partial \beta ^{\theta
}R_{t}^{1-\theta }}{\partial R_{t}}=\left( 1-\theta \right) \frac{\varsigma
_{t}}{R_{t}}.
\end{equation*}%
Thus,

\begin{equation*}
\epsilon _{R}^{\varsigma }=\frac{\partial \varsigma _{t}}{\partial R_{t}}%
\frac{R_{t}}{\varsigma _{t}}=\left( 1-\theta \right) .
\end{equation*}%
If we measure things in terms of a price of consumption tomorrow $c_{t+1},$
name it $q_{t}\,\ $the term is equal to $\left( \theta -1\right) $.

\textbf{Intertemporal-Elasticity of Substitution II. }Observe that the
derevitions we have so far imply:

\begin{equation*}
\frac{c_{t}^{i}}{c_{t+1}^{i}}=\Psi \left( \beta ,R_{t}\right)
\end{equation*}%
\begin{equation*}
\left( \frac{c_{t+1}^{i}}{c_{t}^{i}}\right) ^{1/\theta }=\beta R_{t}.
\end{equation*}%
Taking logarithms on both sides yields:

\begin{eqnarray*}
\Delta c_{t+1} &=&\theta \log \left( \beta \right) +\theta \log (R_{t}) \\
&=&\theta \log \left( \beta \right) +\theta r_{t}
\end{eqnarray*}%
where $\Delta $ is the operator for log-difference ($\Delta x_{t}=\log
\left( x_{t+1}\right) -\log \left( x_{t}\right) $).

Then, we have that
\begin{equation*}
\frac{\partial \Delta c_{t+1}}{\partial r_{t}}=\theta .
\end{equation*}%
but since $\Delta c_{t+1}$ and $r_{t}$ are already log-differences:

\begin{equation*}
\theta =\frac{\partial \left( c_{t+1}^{i}/c_{t}^{i}\right) }{\partial \left(
r_{t}\right) }\frac{R_{t}}{\left( c_{t+1}^{i}/c_{t}^{i}\right) }.
\end{equation*}%
Thus, $\theta $ measures how much future consumption varies with the
interest rate. If $\theta <1$, higher rates imply less future consumption.
If $\theta >1,$ higher rates imply less current consumption. If $\theta =1,$
consumption doesn't vary.

\textbf{CES representation.} Let's relate this expression now to the CES
form we studied before. I want to show that the two period model we just
wrote, can be casted as a CES utility. To see this,

\begin{equation*}
u_{i}=\max_{c_{t}^{i}}U\left( c_{t}^{i}\right) +\beta \cdot U\left(
R_{t}\left( W_{t}-c_{t}^{i}\right) \right) .
\end{equation*}%
We perform a monotone transformation.

We need three operations.

\begin{enumerate}
\item Add to both sides $\left( 1+\beta \right) /\left( 1-1/\theta \right) .$%
\begin{equation*}
u_{i}+\left( 1+\beta \right) /\left( 1-1/\theta \right) =\max_{c_{t}^{i}}%
\frac{\left[ c_{t}^{i}\right] ^{1-1/\theta }+\beta \cdot \left( R_{t}\left(
W_{t}-c_{t}^{i}\right) \right) ^{1-1/\theta }}{1-1/\theta }
\end{equation*}

\item Multiply by $\left( 1-1/\theta \right) $
\begin{equation*}
\left( 1-1/\theta \right) u_{i}+\left( 1+\beta \right) =\max_{c_{t}^{i}}
\left[ c_{t}^{i}\right] ^{1-1/\theta }+\beta \cdot \left( R_{t}\left(
W_{t}-c_{t}^{i}\right) \right) ^{1-1/\theta }.
\end{equation*}

\item and then raise to the $1/\left( 1-1/\theta \right) $ power to obtain:%
\begin{equation*}
\left( \left( 1-\frac{1}{\theta }\right) u_{i}+\left( 1+\beta \right)
\right) ^{\frac{1}{1-\frac{1}{\theta }}}=\max_{c_{t}^{i}}\left( \left(
c_{t}^{i}\right) ^{1-\frac{1}{\theta }}+\beta \cdot \left( R_{t}\left(
W_{t}-c_{t}^{i}\right) \right) ^{1-\frac{1}{\theta }}\right) ^{\frac{1}{1-%
\frac{1}{\theta }}}.
\end{equation*}

We are not there yet since we need to have write it in terms of two
commodities:%
\begin{equation*}
\max_{c_{t}^{i}}\left( \left( c_{t}^{i}\right) ^{1-\frac{1}{\theta }}+\beta
\left( c_{t+1}^{i}\right) ^{1-\frac{1}{\theta }}\right) ^{\frac{1}{1-\frac{1%
}{\theta }}}.
\end{equation*}%
However, we need to multiply by some weights that sum up to one. Thus,

\item Multiply both sides by $\omega ^{\frac{1}{\theta }}$ to obtain:
\end{enumerate}

\begin{equation*}
\max_{c_{t}^{i}}\left( \left( \omega \right) ^{\frac{1}{\theta }}\left(
c_{t}^{i}\right) ^{1-\frac{1}{\theta }}+\omega ^{\frac{1}{\theta }}\beta
\left( c_{t+1}^{i}\right) ^{1-\frac{1}{\theta }}\right) ^{\frac{1}{1-\frac{1%
}{\theta }}}
\end{equation*}%
and now we need:

\begin{equation*}
\omega ^{\frac{1}{\theta }}\beta =\left( 1-\omega \right) ^{\frac{1}{\theta }%
}.
\end{equation*}%
Thus,

\begin{equation*}
\omega =\frac{1}{1+\beta ^{\theta }}.
\end{equation*}%
Thus, the solution equals:

\begin{equation*}
v_{i}=\left( \frac{1}{1+\beta ^{\theta }}\right) ^{\frac{1}{\theta }}\left(
\left( 1-\frac{1}{\theta }\right) u_{i}+\left( 1+\beta \right) \right) ^{%
\frac{1}{1-\frac{1}{\theta }}}.
\end{equation*}

\textbf{Risk-Aversion.} Now assume that the interest rate is actually random
variable $R\left( s\right) $ where s is an underlying state. Let's exploit a
represenation of two-period model and use $\mathcal{CE}_{t}$ to express the
random rates as the certainty equivalent of a single rate. We would obtain
the following utility represenation:

\begin{eqnarray*}
&&U\left( c_{t}^{i}\right) +\beta U\Upsilon ^{-1}\left( \mathbb{E}%
_{t}\Upsilon \left( U^{-1}(U\left( R\left( s\right) \left(
W_{t}-c_{t}^{i}\right) \right) \right) \right) \\
&=&U\left( c_{t}^{i}\right) +\beta U\Upsilon ^{-1}\left( \mathbb{E}%
_{t}\Upsilon \left( U^{-1}(U\left( R\left( s\right) \left(
W_{t}-c_{t}^{i}\right) \right) \right) \right) \\
&=&U\left( c_{t}^{i}\right) +\beta U\Upsilon ^{-1}\left( \mathbb{E}%
_{t}\Upsilon \left( U^{-1}(U\left( R\left( s\right) \right) \right) \right)
\\
&=&U\left( c_{t}^{i}\right) +\beta U\left( \left( W_{t}-c_{t}^{i}\right)
\Upsilon ^{-1}\mathbb{E}_{t}\Upsilon \left( R\left( s\right) \right) \right)
\\
&=&U\left( c_{t}^{i}\right) +\beta U\left( \left( W_{t}-c_{t}^{i}\right)
\underbrace{\mathbb{CE}_{t}\left( R\left( s\right) \right) }_{\text{%
Certainty-equivalent Return}}\right)
\end{eqnarray*}%
So in the two period model, the $\gamma $ controls how we penalize risk: how
we transform random returns into a certainty equivalent interest rate. Note
that we cannot do this separition with CRRA because the two parameters take
the same value. A complication emerges when a model features dynamics and
there is aggregate risk ---if it affects prices at t+1 in consumption at
t+1, ---here that risk is not present because there are no future periods.
We will return to that point later.

\section{Gorman's Aggregation Result}

Take a collection if $i\in \mathcal{I}$ households. Then, the economy admits
a representative household if the conditions of the following theorem hold.

We have the following Theorem:

\textbf{Theorem.} An economy admits a representative agent if the problem of
the Household if and only if their preferences have an indirect utility
representation of the form:

\begin{equation*}
u^{i}\left( W,p\right) =\alpha ^{i}\left( p\right) +v\left( p\right) w.
\end{equation*}

This condition implies that the household's expenditure rule satifies linear
Engle curves. In turn, all preference specifications that lead to linear
Engle curves, admit a representative agent.

Pollack gives a class of utility representations that are the only classes
that lead to Engle curves. What we do in this notes is alter bliss points
for endogenous objects and alter the results to admit more complicated, but
linear, budget sets.

\textbf{References.} See Gorman's Original paper, as well as Pollack 1971.
Also Acemoglu contains an excellent treatment.

\section{Aggregation with Portfolio Constraints}

Let's consider a generic problem. Preferences of agent $i$ are given by:

\begin{equation*}
u_{t}^{i}=U\left( c_{t}^{i}\right) +\beta \cdot U\left( \mathbb{CE}%
_{t}\left( U^{-1}\left( u_{t+1}^{i}\right) \right) \right)
\end{equation*}%
The budget constraint of this agent is the following:

\begin{equation}
c_{t}+\sum_{j\in \mathcal{J}}a_{t+1}^{j}=\underbrace{\sum_{j\in \mathcal{J}%
}R_{t}^{j}\left( s^{t}\right) a_{t}^{j}}_{\equiv W_{t}}  \label{E_BC}
\end{equation}%
where $\mathcal{J}$ is a set of securities with random payoff $%
R_{t}^{j}\left( s_{t}\right) $. In addition, we assume the following
non-negativity conditions:

\begin{equation}
\Gamma _{t}\hat{a}_{t+1}\leq \eta _{t}\left( W_{t}-c_{t}\right)
\label{E_ConstraintSet}
\end{equation}%
where, $\Gamma _{t}$ is a matrix and $\eta _{t}$ a vector. The constraint
set give by $\left( \ref{E_ConstraintSet}\right) $ imposes constraints on
the portfolio holdings of the individual. Thus, the agent maximizes his
utility subject to (\ref{E_BC}) and (\ref{E_ConstraintSet}).

I use $\mathbf{\hat{a}}_{t}$ and $\mathbf{\hat{R}}_{t}\left( s^{t}\right) $
as the vector notation to express the vector of asset holdigns and returns.
We want to show that an economy with this preferences and budget constraints
admits aggregation, that is, that is has a representative agent. The rest of
the problems we study will be adapted to fit into this form, although they
will feature different financial frictions. The gist of the solutions is to
find a parsimonious way to express the in the way I did above. We will prove
the following proposition:

\begin{proposition}
Optimal Consumer allocations are given by:%
\begin{eqnarray*}
c_{t}^{i}\left( s^{t}\right)  &=&\varsigma _{t}\left( s^{t}\right)
W_{t}^{i}\left( s^{t}\right)  \\
a_{t+1}^{j}\left( s^{t}\right)  &=&\left( 1-\varsigma _{t}\left(
s^{t}\right) \right) \phi _{t}^{j}\left( s^{t}\right) W_{t}^{i}\left(
s^{t}\right)
\end{eqnarray*}%
where $W_{t}^{i}$ is defined above, and the following recursion is obtained:%
\begin{equation*}
\varsigma _{t}=\frac{1}{1+\beta ^{\theta }\Omega _{t}^{1-\theta }\left(
\varsigma _{t+1}\right) }
\end{equation*}%
\begin{eqnarray*}
\Omega _{t} &=&\Omega \left( s^{t}\right) =\max_{\mathbf{\hat{\phi}}}\mathbb{%
CE}_{t}\left[ \varsigma _{t+1}^{\frac{1}{1-\theta }}\left( s^{t+1}\right)
<R_{t}^{j}\left( s^{t+1}\right) \cdot \phi ^{j}>|s^{t}\right]  \\
\mathbf{\hat{\phi}}_{t}\left( s^{t}\right)  &=&\mathbf{\hat{\phi}}\left(
s^{t}\right) =\arg \max_{\mathbf{\hat{\phi}}}\mathbb{CE}_{t}\left[ \varsigma
_{t+1}^{\frac{1}{1-\theta }}\left( s^{t+1}\right) <R_{t}^{j}\left(
s^{t+1}\right) \cdot \phi ^{j}>|s^{t}\right]  \\
\text{subject to} &\text{:}&\sum_{j}\phi ^{j}=1;\text{ }\Gamma _{t}\phi
^{j}\leq \eta _{t}.
\end{eqnarray*}
\end{proposition}

\textbf{Proof.} Let's now prove this result. Fix a given period t and
historyt $s^{t}$. The agent solves:

\begin{equation*}
u_{t}^{i}\left( W_{t},s^{t}\right) =\max_{c_{t}^{i},\hat{\phi}_{t}}U\left(
c_{t}^{i}\right) +\beta \cdot U\left( \mathbb{CE}_{t}\left( U^{-1}\left(
u_{t+1}^{i}\right) \right) \right)
\end{equation*}%
subject to :

\begin{equation*}
c_{t}+\sum_{j\in \mathcal{J}}a_{t+1}^{j}=W_{t}\text{ and }\Gamma _{t}\mathbf{%
\hat{a}}_{t+1}\leq \eta _{t}\left( W_{t}-c_{t}\right) .
\end{equation*}

Let's perform a change of variables and express every control variable as a
proportion of wealth:%
\begin{equation*}
c_{t}=\varsigma W_{t}
\end{equation*}%
\begin{equation*}
a_{t}^{j}=\left( 1-\varsigma \right) \phi ^{j}W_{t}
\end{equation*}%
for any arbitrary $\left( \varsigma ,\left\{ \phi ^{j}\right\} _{j\in
\mathcal{J}}\right) .$ Then, the problem can be restated as:

\begin{equation*}
u_{t}^{i}\left( W_{t},s^{t}\right) =\max_{\varsigma ,\phi ^{j}}U\left(
\varsigma W_{t}\right) +\beta \cdot U\left( \mathbb{CE}_{t}\left(
U^{-1}\left( u_{t+1}^{i}\right) \right) \right)
\end{equation*}%
subject to :

\begin{equation*}
\varsigma W_{t}+\sum_{j\in \mathcal{J}}\left( 1-\varsigma \right) \phi
^{j}W_{t}=W_{t}\text{ and }\Gamma _{t}\left( 1-\varsigma \right) \phi
^{j}W_{t}\leq \eta _{t}W_{t}.
\end{equation*}

Observe that at $s^{t+1},$ the value of $W_{t+1}$ will be:

\begin{equation*}
W_{t+1}=\left( 1-\varsigma \right) \left\langle \hat{R}_{t+1}\left(
s^{t+1}\right) \cdot \phi ^{j}\right\rangle W_{t}.
\end{equation*}%
Now, conjecture that $u_{t}^{i}\left( W_{t},s^{t}\right) =U\left(
v_{t}\left( s^{t}\right) \right) W_{t}^{1-\frac{1}{\theta }}.$ If this guess
is true, it must hold for every $t,$ $s^{t}$ and in particular:

\begin{eqnarray*}
u_{t+1}^{i}\left( W_{t+1},s^{t+1}\right) &=&U\left( v_{t+1}\left(
s^{t+1}\right) \right) W_{t+1}^{1-\frac{1}{\theta }} \\
&=&U\left( v_{t+1}\left( s^{t+1}\right) \right) \left[ \left( 1-\varsigma
\right) \left\langle \hat{R}_{t+1}\left( s^{t+1}\right) \cdot \hat{\phi}%
\right\rangle W_{t}\right] ^{1-\frac{1}{\theta }}
\end{eqnarray*}%
at arbitrary values $\left( \varsigma ,\hat{\phi}\right) $ chosen at $t$.
Then this means that the objective in $u_{t}^{i}\left( W_{t},s^{t}\right) $
must also equal:

\begin{eqnarray*}
&&\max_{\varsigma ,\hat{\phi}}U\left( \varsigma \right) W_{t}^{1-\frac{1}{%
\theta }}+\beta U\left( \mathbb{CE}_{t}\left( U^{-1}\left( \left(
v_{t+1}\left( s^{t+1}\right) \left( 1-\varsigma \right) \left\langle \hat{R}%
_{t+1}\left( s^{t+1}\right) \cdot \hat{\phi}\right\rangle W_{t}\right)
\right) \right) \right) \\
&=&\max_{\varsigma ,\hat{\phi}}U\left( \varsigma \right) W_{t}^{1-\frac{1}{%
\theta }}+\beta W_{t}^{1-\frac{1}{\theta }}U\left( \mathbb{CE}_{t}\left(
U^{-1}\left( v_{t+1}\left( s^{t+1}\right) \left( \left( 1-\varsigma \right)
\left\langle \hat{R}_{t+1}\left( s^{t+1}\right) \cdot \hat{\phi}%
\right\rangle \right) \right) \right) \right) \\
&=&W_{t}^{1-\frac{1}{\theta }}\max_{\varsigma ,\hat{\phi}}U\left( \varsigma
\right) +\beta U\left( \mathbb{CE}_{t}\left( U^{-1}\left( \left(
v_{t+1}\left( s^{t+1}\right) \left( 1-\varsigma \right) \left\langle \hat{R}%
_{t+1}\left( s^{t+1}\right) \cdot \hat{\phi}\right\rangle \right) \right)
\right) \right)
\end{eqnarray*}%
which indeed shows that the value function is homogeneous of degree $1-\frac{%
1}{\theta }.$ Moreover,

\begin{equation*}
U\left( v_{t+1}\left( s^{t}\right) \right) =\max_{\varsigma ,\hat{\phi}%
}U\left( \varsigma \right) +\beta U\left( \mathbb{CE}_{t}\left( U^{-1}\left(
U\left( v_{t+1}\left( s^{t+1}\right) \left( 1-\varsigma \right) \left\langle
\hat{R}_{t+1}\left( s^{t+1}\right) \cdot \hat{\phi}\right\rangle \right)
\right) \right) \right)
\end{equation*}%
subject to:

\begin{equation*}
\varsigma +\sum_{j\in \mathcal{J}}\left( 1-\varsigma \right) \phi ^{j}=1%
\text{ and }\Gamma _{t}\left( 1-\varsigma \right) \hat{\phi}\leq \eta
_{t}\left( 1-\varsigma \right) .
\end{equation*}%
Observe that $\varsigma +\sum_{j\in \mathcal{J}}\left( 1-\varsigma \right)
\phi ^{j}=1$ also impies that $\sum_{j\in \mathcal{J}}\phi ^{j}=1.$ Since
the Innada conditions imply $\varsigma >0,$ and $\sum_{j\in \mathcal{J}%
}\left( 1-\varsigma \right) \phi ^{j}>0,$ $\varsigma \in \left( 0,1\right) ,$
and we can ignore $\varsigma $ from the condition above. To aid our
calculations, define the operator defined as the composition of utility
operators $\aleph \equiv U\Upsilon ^{-1}.$ Then,
\begin{equation*}
\aleph ^{\prime }\left( \left( 1-\varsigma \right) \omega _{t+1}\right)
\left( 1-\varsigma \right) ^{-\gamma }\mathbb{E}_{t}\left[ \left( \omega
_{t+1}\right) ^{-\gamma }\left( 1-\varsigma \right) R_{t+1}^{j}\right]
=\left( 1-\varsigma \right) \lambda _{t}+\left( 1-\varsigma \right)
\sum_{i\in 1:DIM\left( \Gamma \right) }\Gamma _{t}^{\left\{ ij\right\} }\mu
_{t}^{i}
\end{equation*}%
where $v_{t+1}\left( s^{t+1}\right) \left\langle \hat{R}_{t+1}\left(
s^{t+1}\right) \cdot \hat{\phi}\right\rangle =\omega _{t+1}$ and $\lambda
_{t}$ and $\mu _{t}^{i}$ are correspondingly the Lagrangeans of the budget
constraint and the KKT multiplier associated with restrcition i in the
matrix $\Gamma _{t}$.

Thus, fix asset 1. Then, excess returns are defined as:

\begin{equation*}
\mathbb{E}_{t}\left[ \left( \omega _{t+1}\right) ^{-\gamma }v_{t+1}\left(
s^{t+1}\right) \left( R_{t+1}^{j}-R_{t+1}^{1}\right) \right] =\sum_{i\in
1:DIM\left( \Gamma \right) }\left( \Gamma _{t}^{\left\{ ij\right\} }-\Gamma
_{t}^{\left\{ i1\right\} }\right) \hat{\mu}_{t}^{i}.
\end{equation*}%
where $\hat{\mu}_{t}^{i}$ are the rescaled KKT multipliers. Note that this
are the exact same first order conditions associated with the following
portfolio problem:

\begin{equation*}
\Omega _{t}=\max_{\hat{\phi}}\Upsilon ^{-1}\left( \mathbb{E}_{t}\Upsilon
\left( v_{t+1}\left( s^{t+1}\right) \left\langle \hat{R}_{t+1}\left(
s^{t+1}\right) \cdot \hat{\phi}\right\rangle \right) \right)
\end{equation*}%
subject to:

\begin{equation*}
\sum_{j\in \mathcal{J}}\phi ^{j}=1\text{ and }\Gamma _{t}\hat{\phi}\leq \eta
_{t}
\end{equation*}

Although $\Upsilon ^{-1}$ is monotone, one has to be careful to have that
term because CRRA utility changes sign $\gamma =1$. In other words, the
operator has discontinuities is $\gamma $ ---fix a sequence of $\gamma $ at
a particular value of $x$. Now, observe that the value function is therfore
equivalent to:

\begin{equation*}
U\left( v_{t+1}\left( s^{t}\right) \right) =\max_{\varsigma }U\left(
\varsigma \right) +\beta U\left( \left( 1-\varsigma \right) \Omega
_{t}\right)
\end{equation*}

Therefore, taking first order conditions yields:

\begin{equation*}
\varsigma _{t}^{-\frac{1}{\theta }}=\beta \left( 1-\varsigma _{t}\right) ^{-%
\frac{1}{\theta }}\Omega _{t}^{1-\frac{1}{\theta }}
\end{equation*}%
or otherwise:

\begin{eqnarray*}
\left( \frac{1-\varsigma _{t}}{\varsigma _{t}}\right) ^{\frac{1}{\theta }}
&=&\beta \Omega _{t}^{\frac{\theta -1}{\theta }}\rightarrow \\
\frac{1-\varsigma _{t}}{\varsigma _{t}} &=&\beta ^{\theta }\Omega
_{t}^{\theta -1}
\end{eqnarray*}%
meaning that
\begin{equation*}
\frac{1}{\varsigma _{t}}=1+\beta ^{\theta }\Omega _{t}^{\theta -1}.
\end{equation*}

Now, we must find a relationship between $v_{t}$ and $\varsigma _{t}$. The
envelope condition implies:%
\begin{equation*}
U^{\prime }\left( W_{t},s^{t}\right) =U^{\prime }\left( c_{t}\right) .
\end{equation*}%
Substituting our guesses for both policies, yields an equivalence:%
\begin{equation*}
v_{t}^{1-\frac{1}{\theta }}\left( W_{t}\right) ^{-\frac{1}{\theta }}=\left[
\varsigma _{t}W_{t}\right] ^{-\frac{1}{\theta }}
\end{equation*}%
Which provides us with the following relationship:%
\begin{equation*}
v_{t}^{1-\frac{1}{\theta }}=\varsigma _{t}^{-\frac{1}{\theta }}\rightarrow
v_{t}=\varsigma _{t}^{\frac{1}{1-\theta }}
\end{equation*}%
This relationship must hold for every $t$ which implies that :

\begin{equation*}
\varsigma _{t}=\frac{1}{1+\beta ^{\theta }\Omega _{t}^{\theta -1}\left(
\varsigma _{t+1}\right) }
\end{equation*}

Where $\Omega _{t}^{\theta -1}\left( \varsigma _{t+1}\right) $ makes the
dependence of $\Omega _{t}^{\theta -1}$ in $\varsigma _{t+1}$ explicit. This
recursion, however, implies that we can solve the model employing recursive
methods to solve for $\varsigma _{t+1}$. Some of the algorithms I describe
next, employ that condition.

Assume $\varsigma _{t}=\varsigma $ as in the case where marginal
propensities are constant. Then,

\begin{equation*}
\frac{1}{\varsigma }=1+\beta ^{\theta }\Omega _{t}^{\theta -1}
\end{equation*}%
and

\begin{equation*}
\Omega _{t}=\underbrace{\mathbb{CE}_{t}\left( \left\langle \hat{R}%
_{t+1}\left( s^{t+1}\right) \cdot \hat{\phi}\right\rangle \right) }_{\rho
_{t}}\varsigma ^{\frac{1}{1-\theta }}
\end{equation*}%
so:

\begin{eqnarray*}
\frac{1}{\varsigma } &=&1+\beta ^{\theta }\frac{1}{\varsigma }\rho
_{t}^{1-\theta }\rightarrow \\
\frac{1}{\varsigma } &=&\frac{1}{1-\beta ^{\theta }\rho _{t}^{1-\theta }}%
\rightarrow \\
\varsigma &=&1-\beta ^{\theta }\rho _{t}^{1-\theta }
\end{eqnarray*}%
One can guess and verify that for the case where $\theta =1,$ log uitility, $%
\varsigma =\left( 1-\beta \right) .$

\begin{itemize}
\item When there aggregate risk, as in the following example, we can again
obtain a similar recursion.
\end{itemize}

\section{Application 1: Angeleto's Idiosyncratic-Risk Insurance Model}

The consumer's problem is written recursively with Epstein and Zin's
preferences. In this economy, there are only two assets: capital that bears
idiosyncratic risk and a risk-free bond traded among the agents of this
economy. The household will will face borrowing constraints given by his
lifetime discounted labor earnings:

\begin{equation*}
h_{t}=\sum_{j=0}^{\infty }\frac{w_{t+j}}{\prod_{k=0}^{j}R_{t+k}}.
\end{equation*}

From the perspective of the household, one can think of $h_{t}$ as the
net-present value of an endowment $w_{t+j}$ but in Angeleto's framework, $%
w_{t+j}$ is actually an endogenous object, wages. There are no aggregate
shocks, but only idiosyncratic shocks to the capital stock.

Angeletos (2007) assumes that capital risk is idiosyncratic ---that is, that
agents do no trade claims against the idiosyncratic return to their capital.
In particular, each household holds physical capital which it runs on it's
own. Households face an idiosyncratic TFP shock $A_{i}.$ The shock is
assumed to be i.i.d accross time. Other than this, there are no aggregate
shocks. Angeletos is interested in studying the steady state of this
economy. However, the model can be adapted easily to study transitional
dynamics or introduce aggregate shocks.

The recursive representation of the household's problem is:

\begin{equation}
V\left( k,b,A;t\right) =\max_{\left\{ c,k^{\prime },b^{\prime
},l^{h}\right\} }U\left( c\right) +\beta U\Upsilon ^{-1}\left( \int \Upsilon
\left( U^{-1}(V\left( k^{\prime },b,A^{\prime };t\right) \right) \psi \left(
A^{\prime },dA\right) \right)
\end{equation}%
where $A$ is a vector process of exogenous shocks that determines
equilibrium prices. This maximization is subject to:

\begin{equation*}
c+i+b^{\prime }=\left[ A^{i}F\left( k,l\right) -wl^{h}\right] +Rb+w
\end{equation*}%
and in addition:

\begin{equation*}
k^{\prime }=i+\lambda k
\end{equation*}

\begin{equation*}
c\geq 0,k^{\prime }\geq 0,\underbrace{Rb^{\prime }\geq -h^{\prime }}_{
_{\substack{ \text{Can't owe tomorrow}  \\ \text{more than human wealth }
\\ \text{from tomorrow onwards}}}}
\end{equation*}%
$\pi $ and $w$ depend on $A,$ and $t$ in a matter known for the consumer.
Hence, $A$ and $t$ are sufficient to determine equilibrium prices. $\lambda $
here is the net-of-depreciation remaining capital stock. Observe that the
problem doesn't feature a linear budget constraint as we had in the
derivation we performed earlier. However, notice that the agent should only
care about the right-hand side of his budget constraint. Therfore, the
Cobb-Douglas technology will require that the return to capital will be
linear function of capital since:

\begin{eqnarray*}
\max_{l}A\left( k_{i}^{1-\alpha }l^{\alpha }\right) -w_{t}l &=&r\left(
A_{i},w_{t}\right) k_{i} \\
&&\text{where,} \\
r\left( A_{i},w_{t}\right) &=&\max_{l/k}A\left( l/k\right) ^{\alpha
}-w_{t}\left( l/k\right)
\end{eqnarray*}%
Thus, financial wealth is decomposed nicely into the return on capital,
wages and the return on bonds:

\begin{equation*}
W_{t}^{i}\left( A_{i},k_{i},b_{t}^{i}\right) =\left( r\left(
A_{i},w_{t}\right) +\lambda \right) k_{i}+R_{t}b_{t}^{i}+w_{t}.
\end{equation*}%
This is a linear representation as we wanted. We are still not ready to cast
this model into the framework above because of the natural borrowing
constraint. However, we can do this via a simple change of variables. Recall
that the natural borrowing limit can be expressed recursively as:

\begin{equation*}
h=w+\frac{h^{\prime }}{R}
\end{equation*}%
Thus, the budget constraint can be exresses as:

\begin{eqnarray*}
c+k^{^{\prime }}+\left( b^{\prime }+\frac{h^{\prime }}{R}\right) &=&\left[
r\left( A^{i},w\right) +\lambda \right] k+Rb+h. \\
&\equiv &W\left( A^{i},k,b\right)
\end{eqnarray*}%
Now, one can perform a second change of variables, where $x^{\prime
}=b^{\prime }+\frac{h^{\prime }}{R}.$ Here, $x^{\prime }$ is the sum of the
agent's human plus financial wealth, but that doesn't include his physical
wealth. Note that the return to $x^{\prime }$ is R, because the following
period, that the RHS of the budget constraint will feature $Rb^{\prime
}+h^{\prime }=Rx^{\prime }$.

The recursive representation of the household's problem is:

\begin{equation}
V\left( k,b,A;t\right) =\max_{\left\{ c,k,b\right\} }U\left( c\right) +\beta
U\Upsilon ^{-1}\left( \int \Upsilon \left( U^{-1}(V\left( k^{\prime
},b,A^{\prime };t\right) \right) \psi \left( A^{\prime },dA\right) \right)
\label{2}
\end{equation}%
where $A$ is a vector process of exogenous shocks that determines
equilibrium prices. This maximization is subject to:

\begin{equation*}
c+k^{^{\prime }}+x^{\prime }=W\left( A^{i},k,b\right)
\end{equation*}%
and in addition:

\begin{equation*}
c\geq 0,k^{\prime }\geq 0,x^{\prime }\geq 0
\end{equation*}%
$W\left( A^{i}\right) $ depend on $A,$ and $t$ in a matter known for the
consumer. This is a property of the model studied by Angeletos (2007), but
carrying these two states may not be sufficient in general, if in particular
the history of shocks matter, for example in the case of non i.i.d shocks.

\section{Optimal Portfolio}

The following\ Lemma, which I prove next explains that optimal consumption,
investment and bond holdings are all linear in wealth. This is a remarkable
finding in my view, as it simplifies very nicely the General Equilibrium
conditions.

\begin{proposition}
Optimal Consumer allocations are given by the following:%
\begin{eqnarray*}
c_{t}^{i} &=&\varsigma _{t}W_{t}\left( A^{i},k,b\right) \\
k_{t+1}^{i} &=&\left( 1-\varsigma _{t}\right) \phi _{t}W_{t}\left(
A^{i},k,b\right) \\
b_{t+1}^{i} &=&\varsigma _{t}\left( 1-\phi _{t}\right) W_{t}\left(
A^{i},k,b\right) -Rh_{t+1}
\end{eqnarray*}%
where $w_{t}^{i}$ and $h_{t+1}$ are defined above, and the following
recursion is obtained:%
\begin{equation*}
\varsigma _{t}=\frac{1}{\sum_{s\geq 0}\left( \beta ^{s}\right) ^{\theta
}\dprod\limits_{\tau =0}^{s}\rho _{t+s}^{\theta -1}}
\end{equation*}%
\begin{eqnarray*}
\rho _{t} &=&\rho \left( \omega _{t+1},R_{t+1}\right) =\max_{\phi _{t}}%
\mathbb{CE}_{t}\left[ \phi \left( r\left( A_{i},\omega _{t}\right) +\lambda
\right) +\left( 1-\phi \right) R_{t+1}\right] \\
\phi _{t} &=&\phi \left( \omega _{t+1},R_{t+1}\right) =\arg \max_{\phi _{t}}%
\mathbb{CE}_{t}\left[ \phi \left( r\left( A_{i},\omega _{t}\right) +\lambda
\right) +\left( 1-\phi \right) R_{t+1}\right]
\end{eqnarray*}

Moreover, if $\theta =1,$ then, $\varsigma _{t}=\left( 1-\beta \right) .$
\end{proposition}

Then, we have the following Corollary:

\begin{corollary}
At a Steady State:%
\begin{eqnarray*}
\varsigma _{ss} &=&\frac{1}{\sum_{s\geq 0}\left( \beta ^{s}\right) ^{\theta
}\left( \rho _{ss}^{s}\right) ^{\theta -1}} \\
&=&\left( 1-\beta ^{\theta }\rho _{ss}^{\theta -1}\right)
\end{eqnarray*}%
and
\begin{eqnarray*}
\rho _{ss} &=&\rho \left( \omega _{t+1},R_{t+1}\right) =\max_{\phi _{t}}%
\mathbb{CE}_{t}\left[ \phi \left( r\left( A_{i},\omega _{t}\right) +\lambda
\right) +\left( 1-\phi \right) R_{ss}\right] \\
\phi _{ss} &=&\phi \left( \omega _{t+1},R_{t+1}\right) =\arg \max_{\phi _{t}}%
\mathbb{CE}_{t}\left[ \phi \left( r\left( A_{i},\omega _{t}\right) +\lambda
\right) +\left( 1-\phi \right) R_{ss}\right]
\end{eqnarray*}
\end{corollary}

Recall that (\ref{E_BudgetConstraintAng}) takes a very similar form as the
budget constraints we studied so far. Then, in terms of our main proposition:

\begin{eqnarray*}
c_{t}^{i} &=&\varsigma _{t}W_{t}\left( A^{i}\right) \\
k_{t+1}^{i} &=&\left( 1-\varsigma _{t}\right) \phi _{t}W_{t}\left(
A^{i}\right) \\
x_{t} &=&\varsigma _{t}\left( 1-\phi _{t}\right) W_{t}\left( A^{i}\right)
\end{eqnarray*}%
with
\begin{equation*}
\varsigma _{t}=\frac{1}{1+\beta _{t}^{\theta }\Omega _{t}^{\theta -1}\left(
\varsigma _{t+1}\right) }
\end{equation*}%
\begin{eqnarray*}
\Omega _{t}^{\theta -1} &=&\Omega \left( \omega _{t+1},R_{t+1}\right)
=\max_{\phi }\mathbb{CE}_{t}\left[ \left( \varsigma _{t+1}\left(
s_{t+1}\right) \right) ^{\frac{1}{1-\theta }}\left[ \phi \left( r\left(
A_{i},\omega _{t}\right) +\lambda \right) +\left( 1-\phi \right) R_{t+1}%
\right] \right] \\
\phi _{t} &=&\phi \left( \omega _{t+1},R_{t+1}\right) =\arg \max_{\phi }%
\mathbb{CE}_{t}\left[ \left( \varsigma _{t+1}\left( s_{t+1}\right) \right) ^{%
\frac{1}{1-\theta }}\left[ \phi \left( r\left( A_{i},\omega _{t}\right)
+\lambda \right) +\left( 1-\phi \right) R_{t+1}\right] \right]
\end{eqnarray*}%
Now, observe that for the case without aggrgate risk, $\varsigma _{t+1}$ is
independent of the shocks. Therfore,

\begin{equation*}
\Omega _{t}^{\theta -1}=\frac{1}{\varsigma _{t+1}}\rho _{t+1}^{\theta -1}
\end{equation*}%
where
\begin{equation*}
\rho _{t+1}=\max_{\phi }\mathbb{CE}_{t}\left[ \phi \left( r\left(
A_{i},\omega _{t}\right) +\lambda \right) +\left( 1-\phi \right) R_{t+1}%
\right] .
\end{equation*}%
Thus,

\begin{eqnarray*}
\frac{1}{\varsigma _{t}} &=&1+\beta ^{\theta }\rho _{t+1}^{\theta -1}\left(
\frac{1}{\varsigma _{t}}\right) \\
&=&1+\beta _{t}^{\theta }\rho _{t+1}^{\theta -1}+\beta ^{2\theta }\rho
_{t+1}^{\theta -1}\rho _{t+2}^{\theta -1}\left( \frac{1}{\varsigma _{t+2}}%
\right)
\end{eqnarray*}%
and continuing forward:

\begin{eqnarray*}
\frac{1}{\varsigma _{t}} &=&1+\sum_{s\geq 1}\left( \beta ^{s}\right)
^{\theta }\dprod\limits_{\tau =0}^{s}\rho _{t+s}^{\theta -1}\rightarrow \\
\varsigma _{t} &=&\frac{1}{1+\sum_{s\geq 1}\left( \beta ^{s}\right) ^{\theta
}\dprod\limits_{\tau =0}^{s}\rho _{t+s}^{\theta -1}}.
\end{eqnarray*}

\section{Aggregation}

We now use the fact that policies are linear and invariant for all agents to
compute the General Equilibrium allocations. The optimal portfolio will
specify a sort of rigidity: capital is fixed when once uncertainty is
revealed. This is a key feature of the model as it allows to obtain the
allocations only as a function of the aggregate capital state. In the next
section, a variant will be used to solve for the optimal allocations of the
Kiyotaki-Moore model.

Entrepreneurs inherit capital. By homogeneity of the production function,
they will hire labor according to:

\begin{equation*}
n_{t}^{i}\left( A_{t}^{i},w_{t}\right) =\arg \max_{l}\left[ F\left(
1,L,A_{i}\right) -w_{t}L\right]
\end{equation*}%
and the rate of return capital units will be given by:

\begin{equation*}
r_{t}^{i}\left( A_{t}^{i},w_{t}\right) =\max_{l}\left[ F\left(
1,L,A_{i}\right) -w_{t}L\right]
\end{equation*}

Recall that the shocks are i.i.d and therefore, we can aggregate the labor
demand by the following

\begin{equation*}
\bar{n}_{t}\left( w\right) =\int n_{t}^{i}\left( A_{t}^{i},w\right) \psi
(A)dA
\end{equation*}%
and
\begin{equation*}
\underbrace{\bar{r}}_{\text{Average Portfolio Return}}=\int r_{t}^{i}\left(
A_{t}^{i},w_{t}\right) \psi (A)dA
\end{equation*}%
By independence, note that:

\begin{equation*}
N^{d}=\bar{n}_{t}K_{t}
\end{equation*}%
and that total profits:

\begin{equation*}
\Pi _{t}=\bar{r}K_{t}
\end{equation*}%
Labor markets must clear so we have:

\begin{equation*}
1=N^{d}\rightarrow 1=\bar{n}_{t}^{-1}\left( \frac{1}{K_{t}}\right)
\end{equation*}

Then,
\begin{eqnarray*}
K_{t+1}+C_{t} &=&Y_{t}=f\left( K_{t}\right) =\left[ \bar{w}n^{-1}\left(
\frac{1}{K_{t}}\right) +\bar{r}\right] K_{t} \\
B_{t+1} &=&0 \\
C_{t} &=&\varsigma _{t}\left[ f\left( K_{t}\right) +H_{t}\right] \\
\varsigma _{t}^{-1} &=&1+\beta ^{\theta }\rho _{t}^{\theta -1}\varsigma
_{t+1}^{-1} \\
1 &=&n^{-1}\left( \frac{1}{K_{t}}\right) K_{t} \\
H_{t} &=&\frac{H_{t+1}+w_{t+1}}{R_{t+1}}
\end{eqnarray*}%
where $\phi $ and $\rho $ solve the above equations. At Steady State:

\begin{equation*}
\varsigma _{ss}^{-1}=\frac{1}{1-\beta ^{\theta }\rho _{ss}^{\theta -1}},
\end{equation*}%
and

\begin{equation*}
\rho _{ss}=\max_{\phi }\mathbb{CE}_{t}\left[ \phi \left( r\left(
A_{i},\omega _{t}\right) +\lambda \right) +\left( 1-\phi \right) R_{ss}%
\right] .
\end{equation*}

\section{Solving for the Steady State}

\begin{itemize}
\item For any possible steady-state value of $K^{ss},$ find equilibrium wage
by solving:%
\begin{equation*}
n_{t}^{i}\left( A_{t}^{i},w_{t}\right) =\arg \max_{l}\left[ F\left(
1,L,A_{i}\right) -w_{t}L\right]
\end{equation*}%
for each possible $A_{t}$. Then, $w_{t}$ must solve:%
\begin{equation*}
1=\left[ \sum_{A\in \mathbf{A}}\pi \left( A\right) n_{t}^{i}\left(
A_{t}^{i},w_{t}\right) \right] K_{t}^{ss}.
\end{equation*}%
For this, note that $F_{l}\left( 1,L,A_{i}\right) -w_{t}=0\rightarrow \left(
1-\alpha \right) F_{l}\left( 1,L,A_{i}\right) =w.$

\item Then, compute:%
\begin{equation*}
r_{t}^{i}\left( A_{t}^{i},w_{t}\right) =\max_{l}\left[ F\left(
1,L,A_{i}\right) -w_{t}L\right]
\end{equation*}

\begin{itemize}
\item for each possible $A_{t}^{i}.$
\end{itemize}

\item Solve for $R$: Find the value of $R$ that makes $\phi _{t}=1.$
\begin{equation*}
\phi _{t}=\phi \left( \omega _{t+1},R_{t+1}\right) =\arg \max_{\phi _{t}}%
\mathbb{CE}_{t}\left[ \phi r\left( A_{i},\omega _{t}\right) +\left( 1-\phi
\right) R_{t+1}\right]
\end{equation*}

\item Compute $H$ given solution to $R$.

\item Compute $H\left( K_{ss}\right) $ and $W_{t}\left( K_{ss}\right) $.
Compute $\varsigma \left( K_{ss}\right) $

\item Find:%
\begin{equation*}
K_{ss}=\left( 1-\varsigma _{t}\right) W_{t}\left( K_{ss}\right) .
\end{equation*}
\end{itemize}

\end{document}
